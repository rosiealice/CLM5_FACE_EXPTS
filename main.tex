\documentclass[draft,linenumbers]{agujournal}

\draftfalse

\usepackage{hyperref}
\hypersetup{
colorlinks=true,
linkcolor=blue,
filecolor=magenta,
urlcolor=cyan}

\journalname{Journal of Advances in Modeling Earth Systems (JAMES)}

\begin{document}

\authors{Rosie Fisher\affil{1},
Will Wieder\affil{1},
Ben Sanderson\affil{1},
Charlie Koven\affil{2},
Keith Oleson\affil{1},
Chonggang Xu\affil{3},
Ashehad Ali\affil{1},
Josh Fisher\affil{4},
Mingjie Shi\affil{4},
Katie Dagon\affil{1},
Danica Lombardozzi\affil{1},
Anthony Walker\affil{5},
Gordon Bonan\affil{1},
David Lawrence\affil{1}
}

\title{Nitrogen cycling in the CLM5: controls on responses to environmental forcing}
\author{rosie fisher}
\date{April 2018}

\affiliation{1}{National Center for Atmospheric Research, Table Mesa Drive, Boulder, Colorado, USA}

\affiliation{2}{Lawrence Berkeley National Laboratory, Berkeley, California, USA}

\affiliation{3}{Los Alamos National Laboratory, Los Alamos, New Mexico, USA}

\affiliation{4}{NASA Jet Propulsion Laboratory, Pasadena, California, USA}

\affiliation{5}{Oak Ridge National Laboratory, Oak Ridge, Tennessee, USA}

\correspondingauthor{Rosie Fisher}{rfisher@ucar.edu}


\begin{keypoints}

\item A suite of modifications to the Nitrogen cycle representation were added to the Community Land Model, version 5 (CLM5).

\item New features of the N model include prognostic $V_{c,max}$ and $J_{max}$, representation of the carbon economic cost of Nitrogen uptake, switching of N uptake between alternative sources, variable tissue C:N ratios and dynamic leaf resorption of Nitrogen.

\item Here we assess the sensitivity of the model to a suite of parameters pertinent to the cycle of Carbon and Nitrogen. We assess the dependence of the both the model state, and also the responses of the system to higher CO2, Nitrogen and temperature to parameter uncertainty. Model runs are conducted at representative individual sites in temperate, tropical and boreal systems.

\item The model responses to CO2 and N fertilization are primarily linked to the representation of the fraction of plants that can fix Nitrogen, and to the costs of N in the environment.

\item Identification of the primary factors driving ecosystem responses to fertilization both assists in the interpretation of global simulations and in the identification of priorities for fieldwork and process updates.

\item This implies that a critical control in the CLM, and in coupled simulations with the CESM for CMIP are dependant on the representation of Nitrogen fixations and its dynamics.

\end{keypoints}

\section{Introduction}

Simulating the cycling of nutrients, and how nutrient availability impacts ecosystem growth and function, has been repeatedly identified as a crucial element of Earth system models (\cite{piao2013}, \cite{gruber2008}, \cite{wang2009}). In the CMIP5 multi-model intercomparison process, only one model, the Community Land Model v4.0, included an active nitrogen (N) cycle. The CLM4.0 projected that inclusion of N dynamics might significantly limit the ability of the terrestrial biosphere to respond to fertilization by increasing atmospheric carbon dioxide (\cite{friedlingstein2006},\cite{friedlingstein2014}, \cite{arora2013}). Subsequent and parallel to CMIP5, many terrestrial biosphere and land surface models that represent N dynamics using alternative structural assumptions have been developed \cite{wang2007}, \cite{zaehle2010}, \cite{goll2012}, \cite{smith2014}). Nonetheless, substantial uncertainty remains in the land surface modeling community regarding how nutrient limitations should be represented in models (\cite{zaehledalmonech2011}).


Unlike representations of photosynthetic responses to environmental conditions, or radiative transfer through forest canopies, our understanding of how nutrient cycling functions both within plants and in whole ecosystems is not governed by any well-tested and widely accepted paradigms. Detailed model-data synthesis activities, notably those conducted at North American free-air carbon dioxide enrichment (FACE) experiments, have revealed important differences in model behavior emerging in part from the absence of a broadly accepted theoretical framework for construction and parameterization of nutrient cycling processes (\cite{zaehle2014}). Few, if any studies, however, have looked at how CO$_{2}$ or N fertilization are governed by the parameters or structure of a given model.


The Community Land Model is the land surface representation within the Community Earth System Model (CESM, \cite{hurrell2013}). Version 5 of the CLM was recently released to the research community as part of the CESM2.0 release. An overview of model developments is presented by \cite{lawrence2018}, a descriptions of the global Nitrogen cycling by Wieder et al. (in prep), the crop model (Lombardozzi et al. in prep), the land use components (Lawrence P et al. in prep), land surface hydrology (Swenson et al. in prep) and plant hydraulics (Kennedy et al. in prep).


The CLM5 model includes numerous major update of the nitrogen cycling representation subsequent to CLM4.0 and CLM4.5. The introduction of several new dynamical features into the CLM5 has implications for both parameteric uncertainty and responses to environmental forcing. Wieder et al. (in prep) report the global-scale implications of the full model update and note that the CLM( responds more strongly to CO2 and less strongly to N fertilization than its predecessors. In this paper, we describe the new components of the biogeochemical cycling of nitrogen, their impacts upon the model system state and responses to environmental forcing.


\subsection{Model Updates}

The CLM5 nitrogen cycling builds on the developments of the implementation of the CLM4.0 'CN' model (\cite{thornton2007}), and on modifications to the biogeochemical cycling (notably vertically stratified soil decomposition processes) included in the interim CLM4.5 model (\cite{koven2013}, \cite{bonan2012}). The new model integrates three major additional prognostic elements of the Nitrogen cycle, including: \\


1) The `LUNA' model (Leaf Utilization of Nitrogen for Assimilation), which simulates distribution of N between different leaf assimilation processes (\cite{xu2012}, \cite{ali2016}) \\

2) The `FUN' module (Fixation and Uptake of Nitrogen), which simulates the dynamics of Nitrogen acquisition from the environment (\cite{fisher2010fun}, \cite{brzostek2014}, \cite{shi2016}).\\

3) The `FLEXCN' N cycle implementation, which is used here primarily to allow variation in tissue C:N ratios, and is adapted from \cite{ghimire2016}.\\



More details on each of these new components is given below. A full technical description of the CLM5 is an appendix to \cite{lawrence2018} and is also available online at \url{(https://escomp.github.io/ctsm-docs/doc/build/html/tech_note/index.html).}


\subsubsection{LUNA}

Most land surface models predict photosynthetic parameters from leaf nitrogen content (\cite{kattge2009}, \cite{bonan2012}). There is significant evidence, however, that plant photosynthetic capacity can respond to environmental conditions, including CO$_{2}$ fertilization (\cite{ainsworth2007}) and changing temperature (\cite{hikosaka2005}), irradiance (\cite{niinemets1998}) and soil moisture (\cite{keenan2009}). Given the importance of photosynthetic capacity as a key model parameter (\cite{rogers2017}) the use of a static photosynthetic capacity appears to be a poor means of capturing its variation in time and space and responses to climatic change (\cite{walker2017}).


The LUNA model predicts the optimal balance of of $V_{c,max}$ and $J_{max}$ for the prevailing time-averaged environmental conditions. The model determines the rate constants that are consistent with a co-limitation of photosynthesis by both electron capture and carboxylation processes. In this fashion, $V_{c,max}$ is primarily controlled by the N per unit leaf area ($N_{area}$, which is itself a function of the target C/N ratio, the specific leaf area, and the prognostic variation in leaf C/N ratios) and environmental conditions. The capacity of the LUNA model to represent appropriate responses to temperature and CO$_{2}$ was assessed by \cite{xu2012} while the global calibration and the geographical predictions of the model were described by \cite{ali2016}.


\subsubsection{FUN}

Under circumstances where insufficient N exists to match all of the carbon assimilated for a given C:N ratio, the CLM4.0 model reduced the gross photosynthetic flux down to the level at which growth could be supported by the assimilated N. This process occurred after the calculation of stomatal conductance (which is linked to assimilation rates via the Ball-Berry model) and therefore created inconsistencies between C assimilation and water cycling, especially under conditions of N limitation, as discussed extensively in the literature (\cite{medlyn2011}, \cite{bonan2012}, \cite{dekauwe2014}, \cite{walker2014}).


One major issue confronting N cycle models is how to deal with this 'excess' carbon under N limiting conditions (\cite{zaehle2010}, \cite{dekauwe2014}). The FUN model operates on the principle that for the alternative sources of Nitrogen acquisition by plants (active uptake from soil, retranslocation from senescent tissues, and symbiotic N fixation) there is a concurrent cost in terms of carbon. Further, the cost of each of these uptake pathways is variable in both time and space. FUN hypothesizes that plants will take up Nitrogen from pathways which are 'cheaper' to them. Where N is scarce in the environment, carbon that would otherwise have been used for plant growth can be deployed to acquire (pay for) more N. The central element of FUN is a simultaneous equation which assumes, firstly, that ($C_{avail}$), the total carbon available (after maintenance respiration and replenishment of stores are accounted for) is either used for plant growth ($C_{growth}$) or for Nitrogen uptake, ($C_{nuptake}$).


\begin{equation}
C_{growth}=C_{avail}+C_{nuptake}
\end{equation}


Further, the Nitrogen acquired from the environment ($N_{uptake}$) must equal that which is deployed in the growing plant tissues. The carbon used for growth is further reduced by the growth respiration term ($f_{gr}$) which applies only to tissues that are constructed:


\begin{equation}
N_{uptake}=C_{nuptake}/N_{cost} =\frac{C_{growth}}{(CN_{target}/(1.0+f_{gr})}
\end{equation}


Thus the Carbon that is expended on N uptake is determined by:

\begin{equation}
C_{nuptake} =\frac{C_{avail}}{ ( (1.0+f_{gr})*(CN_{target} / N_{cost}) + 1) }
\end{equation}


The average Nitrogen cost ($N_{cost}$) is derived from the simultaneous uptake of N across all uptake streams, while the target CN ratio ($CN_{target}$) is the combined target CN ratio of all the plant tissues, weighted for the size of the different pools. The FUN model is documented \href{https://escomp.github.io/ctsm-docs/doc/build/html/tech_note/FUN/CLM50_Tech_Note_FUN.html}{here}


A further issue addressed by FUN is the preferential use of symbiotic N fixation when soil mineral N concentrations are low (\cite{vitousek2002}). The CLM4.0 and CLM4.5 predict N fixation as a linear function of net primary productivity (\cite{wieder2015} illustrate the impacts of uncertainty in this function) but FUN considers fixation rates to be an emergent property of plant assimilation rates (providing carbon for fixation), the relative costs of environmental N acquisition, and temperature (which exerts a primary control over the enzymatic processes, following \cite{houlton2008}). Fixation is preferred when the costs of N acquisition are higher than that of fixation. Note that while the CLM5 has no representation of biological succession (\cite{fisher2018}), this property nonetheless affects the amount of time taken to achieve a spun-up state from bare ground, in productive areas, N fixation is preferred and soil N builds up much quicker. In low productivity areas, carbon is required to fix N from a 'col start' initial state, impacting the survival of marginal plants; thus higher initial stocks of N are required to allow growth once metabolic and turnover costs have been met in early spin-up.


\subsubsection{FLEXCN}

Several elements of model behaviour that are critical for simulation of plant responses to variation in nitrogen availability emerged from the FACE model data synthesis activity that were not captured by previous versions of the CLM Nitrogen cycle (\cite{zaehle2014}, \cite{medlyn2015using}). For example, changes in the C:N ratio accounted for large changes in ecosystem carbon storage, but these are not captured by the static tissue C:N ratios used in the CLM4.0 and CLM4.5 nitrogen cycle. For this reason, we introduce a flexible tissue C:N ratio into the CLM5.


\cite{ghimire2016} implemented a suite of modifications to the CLM Nitrogen cycling model, including variable tissue C:N ratio, prognostic $V_{c,max}$ (resulting from this dynamic leaf N content) and also an alternative Nitrogen uptake algorithm. In the default configuration of CLM5, the LUNA model predicts $V_{c,max}$ from leaf N content (while also taking into account environmental conditions), and the FUN model represents Nitrogen uptake, the primary element of CLM5 utilized from the \cite{ghimire2016} model is the modification to the allocation scheme that allows tissue C:N ratios to be modified in response to varying supply of Nitrogen (although the alternative schemes implemented by \cite{ghimire2016} are available as options). In this new allocation algorithm, the total Nitrogen supply in each timestep is partitioned between tissues according to their relative 'demand' terms (ascertained from the target C:N ratio and the carbon allocated to the pool) and C:N ratios vary with Nitrogen supply rates. This new allocation code we term the FLEXCN module.


\subsubsection{Adaptation of FUN to flexible CN ratios}

The FUN model was originally conceived with a static tissue CN ratio, and will by default allocated an amount of carbon to N uptake that will exactly track the target CN ratio. In CLM5, one of the aims of the development process was to incorporate a varying CN ratio. To that end, it is necessary to modify the amount of Carbon expended on N uptake (and thus the amount of N required). The degree to which tissue CN content varies as N becomes limiting in unclear in the literature, and N cycle models typically include heuristic representations of tissue N adjustments in light of limitations (including \cite{zaehle2010},\cite{ghimire2016}). Here we include a placeholder algorithm that adapts to 1) the environmental cost of N acquisition and 2) the degree to which the plant is already far from it's target N content. When N is expensive in the environment, less C is spent on uptake, increasing CN ratios where N is scarce. When plants become extremely N limited, however, expenditure increases again to prevent CN ratios becoming unrealistically high. This new algorithm is also documented in the CLM5 technical description ((\url{(https://escomp.github.io/ctsm-docs/doc/build/html/tech_note/index.html)}) and includes three tuning parameters ($flexcn_{a}$, $flexcn_{b}$ \& $flexcn_{c}$) the sensitivity of which is investigated here.


\subsection{Content of this paper}

In this paper, we assess in detail the sensitivity of the CLM5 model structure to its input parameters. We use\\

1. A range of single site experiments \\

2. One-at-a-time and global parameter uncertainty analyses\\

3. CO$_{2}$ fertilization experiments and\\

4. Nitrogen addition experiments.\\

to fully investigate the performance of the model under steady-state and step-change modifications in forcing. Given the large amount of analysis and attention devoted to assessing the impacts of transient CO$_{2}$ and N in the context of global feedback loop analysis, we thus consider it a worthwhile exercise to illustrate how parametric choices in the CLM5 might affect overall behaviour in larger-scale experiments, including the controls on global carbon fertilization `$\beta$' values. We also hope to highlight areas of process and parameter uncertainty that appear to require greater attention in future model development and testing cycles. 

Because of the increased number of prognostic processes with the CLM5, many alternative representation of C:N cycle interactions are possible within the same model structural space, including an effective removal of N fixation capacity (where carbon available for fixation is fixed at 0$\%$ of assimilation) and variation in the degree to which to C:N ratios are flexible. We further investigate the consequences of turning off the LUNA model and returning to the fixed leaf N/$V_{c,max}$ relationship from \cite{ghimire2016}.

\section{Methods}


\subsection{Sites}

For this experiment we utilized six different experimental sites. Two in tropical forests (Caxiuana and Barro Colorado Island) two in temperate forest (the Oak Ridge National Lab - Free Air Carbon Dioxide Enrichment (FACE) experiment and Harvard Forest), one in high altitude forest (Niwot Ridge, Colorado) and one in seasonally dry evergreen forest (Metolius Forest).


\subsection{Simulation Protocol}

We spun up the default version of the model for 400 years at each site, in 'accelerated spin-up' mode (where the turnover of slower cycling carbon and nitrogen pools is accelerated, see (/cite{lawrence}) using pre-industrial CO$_{2}$ concentrations (274ppm) and recycling the available site-level meteorological inputs. 400 years was sufficient to bring all carbon and nitrogen pools into equilibrium. Subsequent to this, parameters were perturbed (see below), and a second spin-up conducted for 180 years at each site, with accelerated mode turned off. Using the state at the end of the perturbed parameter spin-ups, we ran transient simulations for each site and parameter combination starting in 1850 and invoking transient CO$_{2}$ and Nitrogen deposition until the end of the observed meteorological forcing. We ran additional CO$_{2}$ and Nitrogen deposition experiments by increasing the CO$_{2}$ concentration to 550ppm. For the Oak Ridge FACE site, we only increased CO$_{2}$ in the growing season, in line with the experimental protocol applied there and using the observed CO$_{2}$ time-series. For all other sites, we increased CO$_{2}$ to 550ppm year round starting in 1997**. For Nitrogen, we added 5kgN $m^{-2}$/year, also starting in 1997, so that the results can be compared with those reported by Wieder et al. (in prep).


\subsection{Perturbed Physics Ensemble (PPE)}

The CLM5 has a very large number of parameters to which the C and N cycles are potentially sensitive. Here we focused on parameters of particular relevance to the C and N cycles and on those which are newly introduced by the modified N cycle. We choose a larger set of parameters for one-at-a-time analyses, followed by a smaller set of these to which we subject the model to global sensitivity tests. Knowledge of model structure as well as previous investigations of the CLM model informs our choice of parameters. Our principle goal is to identify parameters that might have large impacts on responses to model forcing, as opposed to changes in the base state of the model, since these are the primary determinants of the carbon cycle feedback in the context of the Earth System Model. However, we also report responses of the model state to perturbations. Additional investigations into the parametric sensitivity of the hydrology and surface energy balance in the CLM5 are planned (Dagon et al. in prep) and into the plant hydraulics parameterizations (Kennedy et al in prep), thus, we have not included the full range of those parameters here.

In the first instance, we conduct one-at-a-time (OAAT) perturbations of the parameters around a mean state, to allow a straightforward visualization and interpretation of the results. For each parameter, for each site, we conduct four simulations, with two above and two below the default value (as well as the default simulation).


\subsubsection{Parameter Choices}

The parameters included in our sensitivity analysis (Table \ref{table_parameters}) are: \\

1) \textbf{Specific Leaf Area}, \emph{`SLA'}, which is a critical parameter determining the area of canopy (in m$^{2}$) derived from one gram of leaf biomass. This parameter is known to be one of the primary elements of the leaf economics spectrum (\cite{wright2004}), and is well-characterized by the TRY database (\cite{kattge2011}).\\

2) \textbf{Fine root area per unit leaf area}, \emph{`froot\_leaf'}, the ratio of fine root biomass to leaf area. Unlike in previous versions of the CLM, fine root biomass now affects the capacity of the model to acquire both water and nutrients (via the resistance to water flow and the cost of N acquisition), thus the impact of this parameter is of greater interest than it's previous functionality as a simple 'tax' on plant growth. \\

3) \textbf{Stem to leaf area ratio}, \emph{`stem\_leaf'}, reflects the balance between accumulation of stem and woody biomass, and leaf area. Increases in this parameter decrease the leaf area index achievable for a given productivity, and conversely increase the equilibrium woody biomass predictions.\\

4) \textbf{Nitrogen Costs}, \emph{`n\_costs'}, are a set of parameters that determine the environmental cost of Nitrogen uptake from the soils. These six parameters were previously defined by \cite{brzostek2014}. Here and in CLM5 in general, we maintain the ratios of the parameters as defined by the existing FUN model calibration, but then allow the magnitudes of the parameters to change concomitantly, since the definitions of soil N availability vary between CLM5 and \cite{brzostek2014}. These parameters include \emph{ekc\_active} and \emph{ekn\_active}, the rate constants that relate the cost of ectomycorhhizal uptake to root carbon (ekc) and soil N concentration (ekn), and their counterparts \emph{akc\_active} and \emph{akn\_active} for arbuscular mycorrhizal fungi, and \emph{kc\_nonmyc} and \emph{kn\_nonmyc} for non-mycorrhizal active uptake. \\

5) \textbf{The fraction of carbon that can be used for fixation}, \emph{`'fracfixers'}. This parameter is the fraction of the assimilation (gross primary productivity - autotropihic respiration, or $gpp$ - $R_{auto}$) that can be used for symbiotic N fixation by the FUN model, and is a proxy for the ecosystem-level fractional productivity of Nitrogen fixing plants, in lieu of, for example, a prognostic model of N fixer distributions, or an input map of their abundance, both of which could plausibly be added in future generations of CLM. \\

6) \textbf{The fraction of carbon spent on respiration per unit of new tissue (growth respiration)}, \emph{grperc}, was revised substantially in CLM5 based on Atkin O. (pers comm.) from 0.3 to 0.11. Here we test the impacts of this change on overall carbon budgets.\\

7) \textbf{Leaf carbon:nitrogen target}, \emph{leafcn}. CLM5 uses the concept of a 'target' leafcn ratio, around which flexibility is allowed, given the environmental costs of Nitrogen. LeafCN ratios are used also as input to the LUNA model to determine photosynthetic rate constants ($V_{c,max}$ and $J_{max}$) \\

8) \textbf{Slope of stomatal conductance model}, \emph{medlyn\_slope}. This new parameter, which replaces the fixed Ball-Berry slope used in CLM5, (and is PFT-dependant), is critically linked to the response of ecosystems to CO$_{2}$, assessed in this set of experiments. \\

9) \textbf{Respiration intercept}, \emph{lmr\_intercept}. The new leaf maintenance respiration model, proposed by \cite{atkin2015} and adopted by CLM5, has it's intercept as a PFT dependant parameter, with variation between PFTs reported by \cite{atkin2015} and used herein. Here we test the impacts of the uncertainty in the observations of this parameter. \\

10) \textbf{Fraction of uptake that is ectomycorrhizal}, \emph{frac\_ecto}, is another ecosystem-level property defined by the FUN model and allowing the proportions of the fungal symbiotic pathways to be modified. Default parameters are as provided by \cite{shi2016}. This capacity was not fully explored in the CLM5 development process and so here we assess its importance. \\

11,12,13) \textbf{Parameters a,b and c of the flexible CN model linked to FUN}, \emph{cn\_flex\_a, cn\_flex\_b, \& cn\_flex\_c}, as described in Appendix 1, which determine the response of tissue CN ratios to depletion of N in both the environment and the plant tissues themselves.


\subsubsection{Parameters not used}

We conducted a set of tests on the parameters of the denitrification model, to assess the impacts of differing rates of N loss on the system. However, recent analysis has shown that the rates of overall denitrification in CLM5 are very low, overall (compared to plant uptake and immobilization \textbf{I've forgotten why this is? Can someone remind me}) and hence these parameters had very little impact. We did not conduct tests on leaf lifespan, since for evergreen trees in CLM5, leaf lifespan and specific leaf area have very similar impacts on overall achievable leaf area, and for deciduous trees, leaf lifespan is primarily determined by phenological triggers. We did not investigate any parameters relating to the crop model (Lombardozzi et al. in prep) nor to plant harvest, fire, dust, soil biogeochemical cycling (the latter being is largely unaltered from CLM4.5, \cite{koven2013} apart from modifications presented by \cite{lawrence}.


\subsubsection{Parameter Ranges}

\emph{Specific Leaf Area} and \emph{leafcn} are defined in the TRY database, and their distributions are log-normally distributed. Thus, we take the average log-normal standard deviation range from TRY and apply it across all PFTs. Thus the lower bounds are closer to the mean than the higher bounds (Table \ref{table_ranges}). The range of \emph{fracfixers} and \emph{frac\_ecto} were set to vary across the whole logical range from 0 to 1 (0 to 100\% of ecosystem productivity being available for fixation or ectomycorrhizal uptake). \emph{grperc} ranges from its logical lower bound of zero up to the range of the previously used CLM value (0.3). \emph{lmr\_intercept} and \emph{medlyn\_slope} are both set such that the range of values tested represents the range across all the PFTs reported in \cite{atkin2015} and \cite{dekauwe2015}. The ranges of the \emph{cn\_flex} parameters are unknown, since these physiological controls are poorly understood, thus we used ranges that span the response limits of these parameters, particularly for \emph{cn\_flex\_c} which we explored across four orders of magnitude to investigate the impacts of leafCN flexibility. The allometric parameters ***. Nitrogen parameters.


\subsection{Analysis}\\
We calculated the impact of the CO$_{2}$ and N deposition by assessing the ratio between the control and the fertilized simulations for the years 2001-2003. For some parameter settings, the constraint on growth placed the system in a state close to death. Addition of N or CO$_{2}$ therefore caused either a very strong recovery of the plants, or no recovery at all where biomass is already zero, leading to occasional large non-linearities in the relationship between parameters and  with  to environmental forcing. Therefore, for those parameter combinations which cause LAI to be less than 80\% of the control simulation value, we exclude the CO$_{2}$ and Nitrogen responses from our plots, so as to focus on the responses of parameter combinations that give realistic initial vegetation conditions.


\section{Results}
The analysis conducted involved four sites, with 13 parameters perturbed to 4 different values, each of those with an increase CO$_{2}$, N and control run. Including the default simulations, this is 672 different run configurations. In this analysis, we look at the parametric responses of five chosen output variables (GPP, NPP, LAI, Leaf N and vegetation carbon), first at the impacts of parameteric variation on the system state and then on the responses to carbon and nitrogen fertilization. 

\subsection{Parametric control over system state}
Figure \ref{NWRstate} shows the one-at-a-time parameter responses of GPP, NPP, LAI, Leaf N content and the carbon spent on N uptake at the Niwot Ridge site. The other site state variation figures are presented in the Supplementary Information (Figures 1-3).

In general, we found that leaf are indicies were often low for these sites, relative to both observations at the sites, and to the magnitude of LAI where gas exchange fluxes are typically known  to saturate as a function of intercepting leaf area (4-5 m$^{2}$ m$^{-2}$. This is a surprising result, given the generally good agreement of globally-driven model output with LAI reported by \cite{lawrence2018}. A seperate set of simulations (not shown here) conducted with higher allocation to leaves (lower stem and root allocation parameters) indicated that, with the resulting higher LAI's, the model was less sensitive to both CO$_{2}$ and N deposition, given the amplifying effect of increasing leaf area increasing fluxes at unsaturated LAIs.  In the default set of simulations the same parameters dominate the responses for differnt variables, given the intrinsic correlation between LAI and GPP, but for saturating LAI conditions, a greater disparity between the dominant parameters is obtained. 

Here we assess the impacts of each parameter on the overall output both at the Niwot Ridge site shown and the other sites in the SI figures. 

Impact of \emph{slatop}\\
At all sites, specific leaf area (\emph{slatop}) has a large first-order impact on LAI,  and on leaf N concentrations (Figure \ref{NR1 stat}), since it is used directly in the calculation of both of these quantities.  Whether this impact on LAI is translated into an impact on GPP depends on the degree of saturated of productivity at the default LAI. 

For GPP, the impact of SLA is always positive, but the degree of it's impact shifts between sites, as it integrates a trade-off between an increase in LAI and the deployment of thinner leaves (lower carbon content per unit area) and subsequent decreases the total N per unit area of leaf reducing photosynthetic capacity 

The impact on NPP however, can be negative (e.g. Metolius), since the CLM5 can generate an under-storey of shaded leaves, and with it the possibility of those leaves being in negative carbon balance.  SLA driven increases in LAI can in principle also reduce NPP for respiratory reasons. At Niwot Ridge, an optimal SLA is observed with respect to NPP, again reflecting the intrisic trade-offs between costs and benefits of thinner leaves. 

Impact of \emph{froot\_leaf}\\
Root:leaf ratio (\emph{froot\_leaf}) a direct effect on LAI, via changes in allocation fractions to different tissues, an indirect effect on NPP, given the modified maintenance costs of the resultant fine root biomass pools, and (newly introduced in CLM5) also affects the uptake of both nitrogen and water via it's impact on N uptake costs and hydraulic resistance terms respectively. The first two impacts increase NPP with \emph{froot\_leaf} and the latter two decrease it.  Both types of effect are seen here, with the Niwot Ridge and Metolius site suffering very large reductions in GPP, NPP, and LAI at low \emph{froot\_leaf} ratios, but the Harvard forest and Caxiuana sites showing increases. 


Impact of \emph{stem\_leaf}\\
Low values of stem:leaf ratio typically resulted in much higher LAI at all sites, as well as declines in vegetation carbon, as expected from the lower stem carbon allocation. At Niwot Ridge, there is an optimum stem carbon observed, but this is due to the impacts of a too-high leaf area index, which actually appears to cause declines in NPP when LAI is above the default. At all other sites, the impact of increasing leaf allocation is to increase productivity. 

Decreases in stem:leaf ratio (stem\_leaf) below their default condition also caused a decline in leaf N per unit area (figure **e).  The causes of this behaviour are still not understood **

Impact of \emph{Ncosts}\\
The first-order impact of N$_{costs}$ is to increase the respiration losses associated with nitrogen uptake.  Across the range tested here (bearing in mind that these parameters are somewhat unconstrained) there was limited impact of N$_{costs}$ on the system state with the exception of the Niwot Ridge site, which showed a decline in LAI, NPP, GPP and vegetation carbon when costs were high, as well as slight increases in leaf N content where costs were low. Other sites showed more muted responses in the vegetation state. 

Impact of \emph{fracfixers}\\
Increasing \emph{Fracfixers} affects the maximum amount of carbon that plants can pay to symbiotically uptake Nitrogen. In all cases here, it has a relatively moderate and positive impact on the overall system state, in particular at the Niwot and Harvard forest sites. As is the case for N$_{costs}$, with high fixing capacity N\_uptake expenditure is lower, N uptake high, and NPP is moderately increased, resulting in increases in LAI and a slight response of GPP to that change.  These systems are only subject to transiently ramping CO$_{2}$$, and are thus closer to equilibrium than the step-change fertilizations, suggesting that the ultimate state of the model N cycle might eventually be unaffected by N fixer fraction. 

Impact of \emph{leafcn}\\
Changing the target leaf C:N ratio \emph{leafcn} over one standard deviation from the mean TRY values had substantial impacts on all the output variables.  As expected, higher \emph{leafcn} values decrease $N_{area}$, decreasing photosynthetic capacity and hence GPP.  In particular, the higher C:N ratios had leafCN and photosynthetic capacities so low that GPP was almost zero at Metolius and Harvard Forest and more than halved at Niwot Ridge.   

For NPP, the corresponding  decreases in respiration rates at high CN somewhat ameliorated this effect, particularly at Caxiuana, which has high ambient temperatures and low carbon use efficiency as a result.  At Niwot, an optimum leaf CN ratio was apparent in the range of the default values, indicating the possibility that other sites might be operating distant from their theoretical optima according to the model, in line with findings in \cite{lawrence} that V${_c,max}$ numbers might be low compared to observations. We do not find much evidence of saturation in the GPP/leafCN relationships, corresponding to a dominance of carboxylation-limited photosynthesis in these simulations. 

In general, for a given change in NPP, the relative change in LAI was lower than for other parametric modification, reflecting the increased spending on nitrogen uptake inherent in the high $N_{area}$ simulations. Interestingly, despite a lower NPP for low CN values at Niwot Ridge, LAI values were higher, for reasons not yet understood ***


Impact of \emph{grperc}.\\
The response to modifications in the growth respiration coefficient in line with first-order expectations, where the higher numbers (which correspond to the previously used value of 0.33 in CLM4.5) resulted in  NPP reductions, corresponding changes in LAI, impacting GPP at the largest values.   The impact varied between sites, reflecting different partitioning of the carbon budget into maintenance respiration, nitrogen uptake and growth. 

Impact of \emph{medlyn\_slope}\\
The change in the stomatal efficiency parameter of the Medlyn 2011 stomatal conductance model most directly impacts GPP, with cascading impacts on all downstream properties. At the Caxiuana site, the very lowest values caused a complete death of the ecosystem. It is unclear why this occured, since the impacts can be direct, on the carbon balance itself (not enough GPP to cover costs), but also on the leaf temperature (via surface energy balance issues) and hence the respiration rates. It is hard to discern the exact reasons, given the feedback inherent between all of these properties, but no impact on the carbon balance is found at intermediate levels.

At Niwot, a clear optimum was found in the Medlyn parameter, indicating that above this level, stomata lose too much water and below it, they do not photosynthesize enough. At Metolius and Harvard forest, only lower value affected the photosynthetic rates, indicating co-limitation by another process, and the absence of impacts of excessive plant transpiration on the system (note that \cite{kennedy} and \cite{dagon} discuss surface energy and plant hydrodynamic functioning of the model in much greater depth). 

Impact of \emph{lmr\_intercept}\\
The variance in the reported intercepts in \cite{atkin2016} also have important consequences for ecosystem productivity, in line with the uncertainties in the growth respiration parameterization. For the high leaf area simulations (now shown **), the impacts of maintenance respiration parameterization were much higher, reflecting the higher amount of respiring tissue in those simulations. At Caxiuana, the warmest site here, the \emph{lmr\_intercept} was one of the most important contributing factors to the NPP, LAI and vegetation carbon states.  

Impact of \emph{frac\_ecto\_fungi}\\
The fraction of ectomycorhizzal vs arbuscular fungi had limited impact here, suggesting that the parameterization of the varying costs between these uptake pathways did not have a large impact on N uptake. This is somewhat surprising, and suggests that this difference is overridden by other parts of the dynamics parameterized in FUN.

Impact of \emph{cn\_flex\_a}\\
\emph{cn\_flex\_a}  is the intercept above which an increase in the cost of N will reduce expenditure on N uptake.  (see \href{https://escomp.github.io/ctsm-docs/doc/build/html/tech_note/FUN/CLM50_Tech_Note_FUN.html#modifications-to-allow-variation-in-c-n-ratios} for details. Thus, low values of \emph{cn\_flex\_a}  results in reductions in expenditure on N uptake when N is too expensive, and therefore on lower tissue N content, and vice-versa.   For the range investigated here, this parameter had limited impact on the system state. 


Impact of \emph{cn\_flex\_b}\\
\emph{cn\_flex\_b} also increases the amount of carbon spent on N uptake when costs are high, but has limited impacts on the system state here. At low values, there is some indication at Niwot Ridge that it might increase LAI and vegetation carbon, indicating a decrease in the N uptake expenditure and in leaf N concentrations.  The limited impact at other sites suggests that Ncosts are rarely high enough to trigger reductions in N expenditure (notwithstanding the point the this algorithm is a placeholder for a more evidence-based depiction of stoichiometric flexibility). 

Impact of \emph{cn\_flex\_c}\\
 Low values of \emph{cn\_flex\_c} affects the degree to whch plants attempte to maintain tissue CN values near to their target when environmental costs are varying. For the state impacts here, the role of this parameter was limited, but for the step-change fertilizations it was much more important, so we will discuss it's role further in the section below. 

\subsection{CO$_{2}$ and Ndep responses}

To illustrate the relationships between N and CO$_{2}$ fertilization, and to reduce the dimensionality of the output, figures \ref{GPP CO2 and N respones 2001}, \ref{NPP CO2 and N respones 2001} and \ref{LAI CO2 and N respones 2001} illustrate the response for GPP, NPP and, LAI respectively. Vegetation carbon and $N_{area}$ are in the supplemental infomation (\ref{VEGC CO2 and N respones 2001}, \ref{Leaf N CO2 and N respones 2001}).  For Niwot Ridge, figures showing the N and CO$_{2}$ responses are also shown (Figure \ref{NR1 ndep} and \ref{NR1 celev}), while the corresponding figures for the other sites are in the SI.

\subsection{Differences across sites}
The responses to CO$_{2}$ and Nitrogen fertilization were substantially different between sites and across parameter ranges. The default range of CO$_{2}$ fertilization impacts on NPP, for example, ranged from 1.22 to 1.66 at Harvard and Caxiuana respectively, with Niwot having a lower range (1.28) and Metolius higher (1.62). For Nitrogen responses, Caxiuana had a negative response (0.96), Niwot had almost no response (0.98) while Harvard and Metolius were positive (1.25 and 1.09 respectively). Impacts on GPP were all positive, from  (albeit lower for the Caxiuana and Niwot sites) through increases in leaf N content, and impacts on LAI were all positive, reflecting lower expenditure on N uptake (1.05 to 2.09). Interestingly, the highest impacts on LAI were at Niwot, which has almost no change in NPP, reflecting a large initial expenditure on nitrogen uptake ameliorated by fertilization.   
 
\subsection{Impact of parameters}
Across the four sites, parametric impacts were very varied, when compared to the relatively consistant impact of parameters on the system state. A majority of parameters show no consistant relationship with CO$_{2}$ or N deposition responses, and the range of their impacts vary over an order of magnitude between sites.  We describe here the major features of how parameteric variation controls the behaviour of the model under these fertilization scenarios.

Impact of \emph{slatop}\\
 

Impact of \emph{froot\_leaf}\\
Despite it’s  impact on the system state,\emph{froot\_leaf} had very little impact on fertilization responses, except in the case of very low values at Metolius, where it increased the CO2 responsiveness and at Harvard, where low values decrease both responses. At Harvard the low root allocation runs had higher initial LAI, and thus the responsiveness of the system to fetilization is lower since the feedback from LAI to GPP is less proncounced. 

Impact of \emph{stem\_leaf}\\}
Low values of \emph{stem\_leaf} ratio caused lower responses to fertilization for GPP and LAI at Caxiuana and Harvard Forest. Initial LAI was larger for these cases for these sites, explaining the decease in fertilization response as above. However, at Niwot Ridge, CO$_{2}$ fertilization was larger with lower \emph{stem\_leaf} ratios. At Niwot, however, a larger control LAI was associated with lower $N_{area}$, GPP and NPP, suggesting a high nitrogen limitation associated with these states. Nonetheless, fertilization responses of GPP here are very high, (x1.8) in response to quite small changes in $N_{area}$ (~2\%)  and LAI (~22\%) (Figure \ref{NPP CO2 and N respones 2001}). Further, Nitrogen increases LAI substantially, buft only has a very limited impact on GPP and NPP at Niwot Ridge, suggesting that the fluxes might be limited by water availability, and thus explaining the very high CO$_{2}$ sensitivity.

Impact of \emph{N\_costs}\\
Given that they impact on the respiration expended on N uptake and the availability of carbon for tissue growth, the impact of Nitrogen costs is most clear for NPP and LAI (Figure \ref{NPP CO2 and N respones 2001}, \ref{LAI CO2 and N respones 2001}}). At Caxiuana, Niwot and Harvard forest, high initial N costs resulted in stimulation of the N fertilization response, as expected from the alleviation of the higher N limitation in these cases.  There was also a slight increase in the CO$_{2}$ fertilization response (maybe on account of extra carbon alleviating some of the N limitation in those systems).  For very low values, at Caxiuana and Niwot, N responses increased, counter-intituvely. At both these sites, at very low N costs, plants appear to make leaves with higher N content (as per the algorithm which allows a higher $N_{area}$ to be achieved at low N cost). However, the total amount of N in the system remains the same with low N costs, and so these cases do not have higher leaf area. When more total N is added, the leaf area can expand, and thus the N concentration declines. 

Impact of \emph{frac\_fixers}\\
While nitrogen uptake costs were important to the fertilization responses, they do not affect the total amount of N in the system, and therefore appeared to play a secondary role to the amount of N fixation available, in many cases.  The fraction of N fixers in the system has a particularly dominant role over the model response to step changes in N and to CO$_{2}$. In line with expectations, for GPP and LAI, all sites displayed a 'trade-off', whereby high N fixer systems responded strongly to CO$_{2}$ but not N, and low fixer systems responded strongly to N but not CO$_{2}$. 

For NPP, this result is muted for the Caxiuana and Metolius sites, and indeed at Caxiuana the low N fixer cases had a very negative (-8\%) NPP response (despite a positive 7.6\% GPP response), on account of the model increasing the N content of the leaves by up to 10\% in the lowest N cost case (Figure \ref{CAX ndep}). Nonetheless, given the lower N uptake costs in the fertilized N experiment, leaf area was substantially increased (up to 14\%) in these cases too. 

At Niwot Ridge and Metolius, the changing fixation capacity had a very large impact on carbon fertilization, but not on nitrogen response, particularly for GPP. At Metolius, this impact persisted, but at Niwot the fixation capacity did have an impact on LAI. Interestingly, many variables increased NPP at this site, but only \emph{fracfixers} increased LAI, suggesting a limitation in the total available N suppressed the impact of all the other case where NPP was increased, and also that water limitation prevented any greater assimilation of carbon. 


Impact of \emph{leafcn}\\
The impact of target \emph{leafcn} on fertilization responses was complex (recalling that this was a significant control over the system state at all sites). t Caxiuana, high values \emph{leafcn} (low leaf  $N_{area}$) had much stronger responses to N fertilization, indicating the dominant impact of the carboxylation capacity at this tropical site.  At Niwot Ridge, low values of \emph{leafcn} (low leaf  $N_{area}$) had much stronger responses to CO$_{2}$ fertilization (the strongest responses at this site for GPP), potentially since the low stomatal conductance at this water-limited site could benefit most from additional CO$_{2}$ where carboxylation capacity was high. 

Impact of \emph{grperc}\\
\emph{grperc}, or the degree to which growth is 'taxed' is constant and not responsive to any of the physiological processes upstream that depend on fertilization. Thus, its effects are all likely due to indirect impacts on the initial state. Two main impacts were observed. At Niwot Ridge low values of \emph{grperc} caused higher responses to CO$_{2}$ fertilization for GPP and NPP, but not LAI. These values corresponded to high initial LAI (on account of larger amounts of C being available for N uptake) and therefore allowed greater leverage of the increasing CO$_{2}$.  At Harvard forest, high \emph{grperc} had caused low levels of LAI, resulting in increased capacity for CO$_{2}$ fertilization at that site.

Impact of \emph{medlyn\_slope}\\
At Niwot Ridge, low values of the Medlyn Slope parameter (with high water use efficiency) were able to respond very strongly to increasing Nitrogen. These cases also had higher GPP in the initial state, and their higher WUE enabled them to take advantage of the increasing N availability.   These cases also ended up with higher vegetation carbon. 

Surpriingly, given its critical role in the responsiveness of stomata to CO$_{2}$, we did not find a widespread impact of the Medlyn parameter on the fertilization reponse, potentially suggesting that the impact on the state condition had already accounted for the impact of this parameter. 

Impact of \emph{lmr\_intercept}\\
In general the impacts of the maintenance respiration intercept were small, and limited to the impact of large values increasing fertilization responses to CO$_{2}$ and N, potentially on account of the low LAI and thus greater feedback between LAI and GPP. 

Impact of \emph{frac\_ectomy\_fungi}\\
In line with its limited impact on the system state, \emph{frac\_ectomy\_fungi} had no discernible impact on fertilization responses. 

Impact of \emph{cn\_flex\_a}, \emph{cn\_flex\_b} and \emph{cn\_flex\_c}\\
As with it's impact on the state, there was limited impact of \emph{cn\_flex\_a} on the fertilization responses, indicating that this parametr is 

Impact of \emph{cn\_flex\_b}
Low values of \emph{cn\_flex\_b}, where the leafCN concentration can increase in response to fertilization, showed higher responses to n fertilization, particularly at Harvard Forest. Small values of  \emph{cn\_flex\_b} should  decrease the amount of carbon spent on N uptake when NCosts are high, thus, a relaxation of the cost of N will increase the amount spent on uptake, and subsequently $N_{leaf}$ and all fluxes downstream.  Thus, these changes result from the relaxation of an over-suppression of leaf N content. 

Impact of \emph{cn\_flex\_c}\\
The impact of \emph{cn\_flex\_c} (the paremter controlling the degree to which plant CN ratios are constrained to their target) appeared to have its strongest impact on the fertilization responses of vegetation carbon at Harvard Forest and Niwot Ridge (and Caxiuana, to a smaller extent). These impacts do not mirror those on leaf area index or NPP. At Harvard Forest and Caxiuana, there is a relative increase in the leaf N content, while at Metolius and Niwot, this parameter has a dominant control over the impacts of leaf N to CO$_{2}$ fertilization.  The impacts on vegetation carbon are likely due to the larger decreases in tissue N content that are permissable when values of this parameter are high, shown clearly at Niwot Ridge and Metolius (e.g. Figure \ref{NR1 celev}, \ref{MET celev}).  Of course, if N concentrations are allowed to decline too much in response to rising CO$_{2}$, then productivity declines, and other complext feedbacks occur, as shown at Harvard Forest and Niwot Ridge (Figure \ref{HRV celev}, \ref{CAX celev}).  In terms of the overall impact, we can see that the degree of flexibility of the tissue CN ratio acts as an orthoganal control on fertilization response to the total fixation capacity.

\section{Discussion}
The overall responses of land surface models to CO$_{2}$ fertilization are the subject of intense investigations, and vast quantities of computing time will be expended on the single release version of each model, under the auspices of the various CMIP model inter-comparison projects and associated activities  (\cite{meehl2014}).  In all of these activities, the carbon-cycle feedbacks reported by each modeling group will be from a single default instance of the parameter space.  At the time of writing, we do not yet know the climate feedback strength of the default version of CLM5 compared to it's predecessors, but it is clear from these investigations that many model parameters can and will have had a large impact on these outcomes.  In particular, as expected, parameters that relate most strongly to the acquisition of N, (\emph{frac\_fixers}, \emph{Ncosts}) and to the flexibility of plant tissues in their C:N ratio (\emph{leafcn}, \emph{cn\_flex\_b}, \emph{cn\_flex\_c}), have dominant roles over the fertilization responses to CO$_{2}$ and nitrogen.  It has long been understood that uncertainties in the representation of Nitrogen are large(\cite{zaehle2014}), but previous model-model comparison studies have struggled to illustrate the exact cause of differences between models predictions, given the complexity of the structural and parametric differences involved. The new CLM5 model has a number of explicit parameterizations, many of which were hard-coded or assumed to be 0 or 1 in previous versions of the model. This allows both a clear exploration of the sensitivities of the parameter space to critical assumptions, as shown here, as well as more comprehensively tuning by future studies and comparisons with relevant datasets. 

In the CMIP6 generation of land surface models, it is expected that a greater number of model submissions will contain active representations of N limitation. We propose here that understanding the model parameterizations and structural decisions that contribute to the relative fertilization responses is critical to understanding the implications of the range of results obtained for all models. 

\subsection{Limitations}
\subsubsection{Limitations of step-change experiments}
Here investigate the parameter space of the model using step-change experiments. However, such experiments necessarily involve unrealistically rapid changes in environmental boundary condition, and cannot necessarily be considered as analagous to either real-world or transient future scenario conditions. Further, we expect that the carbon and nitrogen cycles mayb take some time to adjust to the new conditions, given slow time-scale processes associated with decomposition of litter with change C:N ratios, and feedbacks between increasing vegetation biomass and soil biogeochemistry.

To test the impact of time on the fertilization responses, we also analyzed the impact of fertilization at 5 and 50 years into the elevated CO$_{2}$ and N experiments. At 50 years, the impacts of fertilization on GPP, NPP and LAI were similar, result not shown) to those at 15 years, but at 5 years, in the N limited sites (NWR and HVF), much more significant impacts of N fertilization were apparent x~2.06 impact on NPP, compared to ~0.95 in the 2015 analysis at NVR), indicating that the results described above had already been subject to progressive N limitation and were closer to an equilibrated state.

\subsubsection{Limitations from single site analyses}
The analysis presented here deliberately does not focus on the relative fit of the model to the data collected at the individual sites. The fit at individual sites of the globally adjusted model cannot be seen as indicative or otherwise of its skill in the absence of specific parameterization using the local vegetation, soil and hydrological characteristics, which is beyond the scope of this paper.

Instead, we chose a set of individual sites, rather than global analysis to extensively probe a set of alternative vegetation and climate conditions under realistic forcing scenarios.  We also do not make any assessment of how well the model output fits observations across this parameter space, since the relevant datasets with which to test alternative model configurations are, we argue, the set of globally-relevant data products described by \cite{lawrence2018}. Future studies will consider the impact of alternative parameterizations, informed by these site-level analyses, on the global performance of CLM5, but it is clear that a wide range of biogeochemical responses are possible, within the context of this model structure.  


\subsubsection{Parameter calibration of the default model, or absence thereof}

During the development of CLM5, we conducted extensive testing into the possibility of inverse-calibration of the model using neural network derived emulators, in conjunction with a variety of gridded global data products. We did not use the results of this effort for a variety of reasons, including,  1) the high dimensionality of the space, 2) the degree of non-orthogonal behaviour in the parameter response surfaces,  3) the tendency of the modeled ecosystem to die under past low-CO$_{2}$ conditions when calibrated using present-day CO$_{2}$, 4) the dominance of high productivity areas over the calibration of a particular PFT at the expense of marginal areas, and the resulting pattern that those marginal areas die off in model spin-up, 5) the subjectivity in weighting of the different data products and 6) philosophical issues concerning whether the structural validity of the calibrated model could be independently assessed, given the sparsity of validation datasets. The parameterization used in the default version of the model utilized the mean trait values where possible, from trait datasets (\cite{lawrence2018}) in combination with targeted tuning of parameters   (including allocation parameters and Nitrogen costs)that were less well constrained by data, to avoid excessively spurious values for some model states.

We recognize that this less objective calibration method is than is ideal, and ongoing studies (\cite{dagon}) will focus on more limited scope calibration attempts to overcome the difficulties described above, as illustrated in a recent study by \cite{fer2018}, .  

The development of the CLM5 slightly pre-dated the operational usage of the International Land Model Benchmarking Project (ILAMB) package \cite{collier2016} for assessing model skill across a very broad range of model components.  In principle, integrating more targeted dataset and data products should narrow the range of model parameter sets that are acceptable in comparison with the suite of available and relevant datasets, as well as accelerate the rejection of model structures that are unrealistic. Ideally, this effort should include synthesis of ecosystem manipulation experiments that provide altered boundary conditions in real life, given the large differences observed here between the sets of parameters affecting present day conditions and those that impact transient responses.

\subsubsection{Temperature and water responses}
In this study we investigate the impacts of parametric uncertainty on the carbon and nitrogen responses of the CLM5. In particular, we probe a targeted parameter set comprising those parameters that determine the impacts of CO$_{2}$ and N fertilization. The full response of the model has considerably more dimensions than these two transient properties. In particular, here we have not considered responses to temperature nor to changes in the supply or demand of water. The suite of parameters than in principle control temperature responses of the model are somewhat distinct from the parameters considered here, given the large number of processes that are directly or implictly affected by temperature (photosynthesis, respiration, soil decomposition, N mineralization, cryosphere interaction) and should be the subject of further investigations. Properties more closely linked to water availability are associated with the hydrological components of the model as well as the PHS plant hydraulics scheme and will be investigated further by \cite{dagon} and \cite{kennedy}. 
 

\subsection{The status of the CLM5 representation of plant carbon and nitrogen representation}
Many model development efforts are motivated under the auspices of 'reducing uncertainty'.  Our efforts to improve the fidelity of the nitrogen cycle in the CLM5 are partly motivated by the need to move the model behaviour closer to several key observable features of the system. Notably, the dynamics of leaf Nitrogen content with respect to environmental scarcity (\cite{zaehle2014}, \cite{brzostek2014}), the dynamics of photosynthetic capacity with respect to environmental conditions (\cite{xu2012}, \cite{ali2016},\cite{rogers2017}), the changes in N fixation rate with vegetation state and the scarcity of N (\cite{vitousek2002}), and to include a means of deploying carbon as a means to acquire additional nitrogen. The latter acts as a proxy for the numerous ways in which plant symbiont systems might adapt to low nutrient conditions, including increases in fine root exudation and allocation, fungal biomass, tissue metabolic rates and turnover/nutrient foraging.  We argue that all of these development generate model emergent behaviour that is closer in essence to reality. However, in doing so, we integrate a suite of physiological processes that lack complete mechanistic and quantitative understanding.

In this study, we found numerous instances where the model reduced carbon use efficiency by increasing tissue C:N ratios in response to forcing. In principle, there exists an optimal leaf CN ratio that maximizes leaf C export under a given set of conditions. Despite its optimal allocation of leaf N within the leaf (to photosynths

\subsubsection{Discourse on how target CN ratio could be optimized but isn't}

\cite{thomas2014}, \cite{friedlingstein1999}, \cite{franklin2012}, \cite{mcmurtrie2013}, \cite{anten2011}, \cite{vanwijk2003}


\subsubsection{Discuss geographical spread of predictions for N and CO2 fertilization}

\cite{hickler2008}


\subsubsection{Discourse on whether nitrogen fixation is realistic in CLM5}
The 'fraction of fixers' (\emph{frac\_fixers}) is a proxy for the community level composition of Nitrogen fixing plants. Technically, because plant density is not represented in the CLM5, this parameter really represents the fraction of net assimilated carbon ($gpp$ - $R_{auto}$) that can be used for symbiotic N fixation. Ideally, this feature of the model would be prognostic, and the fraction of N fixing plants would response to changes in environmental conditions and the relative competitive abilities of N fixing plants. In general, Nitrogen fixers are known to primarily exist in early successional parts of ecosystems (references**), and therefore representation of their ecological dynamics would likely include a representation of vegetation demographics and succession (\cite{fisher2018vegetation}, \cite{trugman2016climate}). The results included here indicate that the



\cite{thomas2013global} find that the CLM4.0 model responds very strongly to nutrient additions, and is heavily limited by nitrogen availability in the present day.



\section{Conclusions}




\pagebreak
\begin{table}
\begin{center}
\begin{tabular}{ |c|c|c|c|c| } 
 \hline
 Parameter Name & Meaning & Units & Min & Max \\
  \hline
 slatop & Specific Leaf Area & m2/g \\ 
 froot\_leaf & Fine root:leaf ratio. & g/g \\
 stem\_leaf  & Stem:leaf ratio. & g/g \\ 
 n\_costs    & 6 parameters related to Nitrogen costs & g/g \\
 fracfixers  & Fraction of C availble for fixation & - \\
  grperc  & Fraction of C used for growth respiration  & - \\
  leafcn  & Leaf C:N ratio. & g/g \\
     medlyn\_slope  & Slope of stomatal conductance curve & g/g \\
      lmr\_intercept & intercept of leaf N /respiration relationship &\\
      frac\_ecto & Fraction of active uptake that is ectohycorrhizal & \\
      cn\_flex\_a & Parameter of N flexibilty algorithm& \\
      cn\_flex\_b & Parameter of N flexibilty algorithm & \\
      cn\_flex\_c & Parameter of N flexibilty algorithm & \\
\hline
\end{tabular}
\end{center}
\caption{CLM5 Parameters subject to sensitivity analysis in this study. }
\label{table_parameters}
\end{table}


 \begin{table}
\begin{center}
\begin{tabular}{ |c|c|c|c|c| } 
 \hline
 Parameter Name & s1 &s2 & s3 & s4\\
  \hline
 slatop & 0.6761 & 0.8769 &1.2785 &1.4791\\ 
 froot\_leaf &  0.342 &0.671 &1.05 & 1.1\\
 stem\_leaf  &  0.3057 &0.61 &1 &1\\ 
 n\_costs    & 10$^{-2}$ &10$^{-2}$&  10$^{1}$& 10$^{2}$\\
 fracfixers  & 0 &0.5 & 2 & 4 \\
  grperc  &  0 & 0.5& 2 & 3\\
  leafcn  &0.7413 & 0.8932 & 1.1970  & 1.349\\
  
     medlyn\_slope  &0.5258 & 0.7191 & 1.1057  &1.29\\
      lmr\_intercept & 0.8539 & 0.9531 & 1.1512& 1.25\\
      frac\_ecto & 0 &0.5& 1 & 1 \\
      cn\_flex\_a &0 &0.5 & 2  &4\\
      cn\_flex\_b & 0.1 & 0.5 & 2  & 4\\
      cn\_flex\_c &0.001 & 0.1 & 10 & 100\\
\hline
\end{tabular}
\end{center}
\caption{Multipliers on the original values in CLM5 used in this sensitivity analysis .}
\label{table_ranges}
\end{table}



\section{Figures}

\begin{figure}[h]
     \caption{Sensitivity of system state to parameter perturbation at Niwot Ridge site.}
     \label{NWRstate}
  \end{figure}
Figure 1: Sensitivity analysis of state of CLM at site to N parameters.

Figure 2: Sensitivity analysis of the response of CLM5 to CO2 \& Ndep fertilization @ORNL

Figure 3: Sensitivity analysis of state of CLM at site to N parameters @BCI

Figure 4: Sensitivity analysis of the response of CLM5 to CO2 \& Ndep fertilization @BCI

Figure 5: Sensitivity analysis of state of CLM at site to N parameters @CAX

Figure 6: Sensitivity analysis of the response of CLM5 to CO2 \& Ndep fertilization @CAX

Figure 7: Operation of the flexCN ratio module.

Figure 8: Operation of the leaf N retranslocation module

Figure 9: Details on the N uptake in FUN


\section{To Do}
Add data from the CO2 experiment to figures.

Add a figure of the transient responses through time to CO2 and N... 

Look at the transient response to ramping CO2 and see if its different to the step response

Also add LUNA on and off simulations to figures.

Make a figure with a bunch of extra outputs for the state at Niwot Ridge (CUE, fraction NPP nuptake, BTRAN)
\nocite{*}

Make some SI figures with the high leaf run (state for Niwot?)

Make all on the same figure figure

Slightly re-format the fert figures. (dashes, GPP label)
Check the CN figures. They don't match the NWT figures. 


\bibliography{aguCLM5_bibtex}







\end{document}

