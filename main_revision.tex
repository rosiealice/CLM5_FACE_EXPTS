\documentclass[draft,linenumbers]{agujournal}

\draftfalse

\usepackage{hyperref}
\hypersetup{
colorlinks=true,
linkcolor=blue,
filecolor=magenta,
urlcolor=cyan}

\journalname{Journal of Advances in Modeling Earth Systems (JAMES)}

\begin{document}
%tentative author list...
\authors{Rosie A. Fisher\affil{1},
William R. Wieder\affil{1,2},
Benjamin M. Sanderson\affil{3},
Charles D. Koven\affil{4},
Keith W. Oleson\affil{1},
Chonggang Xu\affil{5},
Joshua Fisher\affil{6},
Mingjie Shi\affil{6},
Anthony P. Walker\affil{7},
David M. Lawrence\affil{1}
}

\title{Parametric controls on vegetation responses to biogeochemical forcing in the CLM5}
\author{Rosie Fisher}
\date{November 2018}

\affiliation{1}{National Center for Atmospheric Research, Table Mesa Drive, Boulder, Colorado, USA}

\affiliation{2}{Institute of Arctic and Alpine Research, University of Colorado, Boulder Colorado, USA}

\affiliation{3}{Centre Europ\'een Research et de Formation Avenc\'ee en Calcul Scientifique, Toulouse, France}

\affiliation{4}{Lawrence Berkeley National Laboratory, Berkeley, California, USA}

\affiliation{5}{Los Alamos National Laboratory, Los Alamos, New Mexico, USA}

\affiliation{6}{Jet Propulsion Laboratory, California Institute of Technology, Pasadena, California, USA}

\affiliation{7}{Oak Ridge National Laboratory, Oak Ridge, Tennessee, USA}

\correspondingauthor{Rosie Fisher}{rfisher@ucar.edu}


\begin{keypoints}
\item The Community Land Model, version 5, contains numerous modifications to representations of the vegetation carbon and nitrogen cycle.

\item The control model state, and it's responses to CO$_{2}$ and nitrogen addition are sensitive to parameter choice, but parameter responses are varied across sites and experiments.

\item Parameters controlling nitrogen fixation dominate fertilization responses. Many other fertilization responses are modified indirectly, via changes in the pre-fertilization state. 
\end{keypoints}

\begin{abstract}
Future projections of land carbon uptake in Earth System Models are affected by land surface model responses to both CO$_{2}$ and nitrogen fertilization. The Community Land Model, version 5, contains a suite of modifications to carbon and nitrogen cycle representation, and globally, has a lower nitrogen response and higher CO$_{2}$ response than it's predecessors. To improve our understanding of the controls over the fertilization responses of the new model, we assess the sensitivity of the model to a suite of thirteen parameters pertinent to the cycling of carbon and nitrogen by vegetation. We assess the parameter dependence of the model state, and also the responses of the system to CO$_{2}$ and nitrogen fertilization at individual sites in temperate, tropical and boreal systems.

 The impact of fertilization is dependant both on parametric variation explored here varies with the balance of limiting factors (water, temperature, nutrients, light) in the pre-fertilization model state.  The model parameters that impact the pre-fertilization state are in general not the same as those that control fertilization responses, meaning that calibration to present-day conditions will not necessarily constrain future projections. Where pre-fertilization state has low leaf area, fertilization induced increases in leaf production amplify the model response to the initial fertilization.  Model responses to CO$_{2}$ and N fertilization are strongly impacted by how much plant communities can increase their fixation of nitrogen.  Further studies focused on how alternative parameters impact biogeochemical function would provide useful compliments to the ongoing CMIP6 process.
\end{abstract}

\section{Introduction}
Simulating the cycling of nutrients, and how nutrient availability impacts ecosystem growth and function, has been repeatedly identified as a crucial element of Earth system models (\cite{piao2013}, \cite{gruber2008}, \cite{wang2009}). In the CMIP5 multi-model inter-comparison, one land model, the Community Land Model v4 (CLM4), included nitrogen (N) cycling and limitation of plant growth. In CLM4 projections, the inclusion of N dynamics significantly limits the ability of the terrestrial biosphere to respond to fertilization by increasing atmospheric carbon dioxide (\cite{friedlingstein2006}, \cite{thornton2007} \cite{friedlingstein2014}, \cite{arora2013}), leading to speculation that the addition of nitrogen limitation as a more general feature of models might reduce the capacity for vegetation to respond to CO2 fertilization and hence reduce the degree to which the terrestrial biosphere can mitigate increasing atmospheric CO$_{2}$. 

The newly-released CLM version 5  (CLM5,\cite{lawrence2018}) includes numerous major updates relative to CLM4 and CLM4.5 (\cite{lawrence2011}). The introduction of several new dynamical features, including substantial modification of the vegetation nitrogen cycling, as well as the introduction of plant hydrodynamics, altered surface hydrology and dynamic crop representation, has implications for responses to environmental forcing. \cite{wieder2019} report the global carbon cycle implications of these changes and note that, in CLM5, carbon uptake responds more strongly to CO$_{2}$ and less strongly to N fertilization than its predecessors, with both fertilization responses being closer to experimental results than the previous versions of the model. 

Given the large amount of analysis and attention devoted to assessing the impacts of transient CO$_{2}$ and N in the context of global biogeochemical feedback analysis, including the controls on global carbon fertilization `$\beta$' values (the change in ecosystem carbon storage in response to CO$^{2}$ fertilization) (\cite{arora2013}, \cite{friedlingstein2014}), we consider the illustration of how parametric choices in the CLM5 might affect and be responsible for overall behaviour in larger-scale simulations to be an important feature of the documentation of the new model.

In this paper, to that end, we describe the new components of the biogeochemical cycling in CLM5, and assess in detail the sensitivity of the CLM5 model parameterization of ecosystem biogeochemistry. We use a range of single site simulations, combined with one-at-a-time parameter uncertainty analyses, and CO$_{2}$ and N fertilization experiments, to investigate the dynamics of the model under steady-state and step-change modifications in forcing. We also hope to highlight areas of process and parameteric uncertainty that appear to require greater attention in future model development and testing cycles.

Model uncertainty quantification efforts are often framed as an attempt to narrow the parameter space to a small range of 'important' parameters, as-determined by their impact on the model's mean-state predictions (e.g. \cite{dietze2014}, \cite{ricciuto2018}).  Predicted global biogeochemical feedbacks, however, are typically related to the response of the system to forcing, and not necessarily to those system properties that control the mean state. Narrowing the range of variables of interest to those that only impact control states and fluxes (and directing more resources to studying those variables) is thus likely to miss important aspects of transient responses under climate change scenarios.  Similarly, much effort is expended on calibration of models to present-day conditions (\cite{fer2018},\cite{lu2018}, when their responses to climate forcing might be relatively unconstrained by these efforts.  Thus, our motivating hypotheses here are:

1. The parameters that dominate responses to CO$_{2}$ and nitrogen fertilization are distinct from those that control the pre-fertilization model state.

2. There is a trade-off between model parameterizations that allow strong CO$_{2}$ fertilization and those that allow strong N fertilization. 

The analysis herein does not focus on the relative fit of the model to the data collected at the individual sites. The fit at individual sites of the globally adjusted model typically cannot be seen as indicative or otherwise of its skill in the absence of specific parameterization using the local vegetation, soil and hydrological characteristics. A focus on model fit at individual sites would preclude detailed consideration of how parameter variation impacts fertilization responses, as presented here. Assessment of the skill of the CLM5 model against a broad suite of globally-relevant data products and manipulation experiments is the focus of \cite{lawrence2018} and \cite{wieder2019}. Future studies will consider the impact of alternative parameterizations, informed by these analyses, on the global performance and skill of CLM5.

\section{Methods}

\subsection{CLM5}
 Version 5 of the CLM was released to the research community in 2018 as part of the CESM2.0 release, and open-source releases and development are housed at \url{https://github.com/ESCOMP/ctsm/}. An overview of model developments is presented within this special issue by \cite{lawrence2018}, a description of the global nitrogen cycling by \cite{wieder2019}/. The plant hydraulics model is documented by \cite{kennedy2019}. Further manuscripts will describe the crop, urban and land use models, sensitivity to forcing and calibration of fast-timescale processes.
 
\subsection{Updates to biogeochemistry Cycling in the CLM5}
The CLM5 biogeochemistry scheme builds on the initial implementation of carbon-nitrogen interactions in CLM4 (\cite{thornton2007}),  and on modifications to the biogeochemical cycling (notably, a vertically stratified soil decomposition processes, and a slowing of the soil mineral N loss pathways) included in the CLM4.5 (\cite{koven2013}, \cite{bonan2012}). 

 A full technical description of the CLM5 is beyond the scope of this paper and is available as an appendix to \cite{lawrence2018} and also online at \url{(https://escomp.github.io/ctsm-docs/doc/build/html/tech_note/index.html)}.

In this paper, we focus on the response to CO$_{2}$ and nitrogen deposition, which are primarily mediated by the biogeochemical processes within the model. The primary changes to the biogeochemical cycling in CLM5 are within the vegetation nitrogen cycle, wherein the model contains three new prognostic elements that are likely to impact responses to fertilization.

Firstly, the  `LUNA' module (Leaf Utilization of Nitrogen for Assimilation),  simulates distribution of N between different leaf assimilation processes (\cite{xu2012}, \cite{ali2016}). Most land surface models predict photosynthetic parameters directly from leaf nitrogen content (\cite{kattge2009}, \cite{bonan2012}). This method, however, appears to be a poor means of capturing variation in photosynthetic capacity in time and space and responses to climatic change (\cite{walker2017}). There is significant evidence that plant photosynthetic capacity can respond to environmental conditions, including CO$_{2}$ fertilization (\cite{ainsworth2007}), changing temperature (\cite{hikosaka2005}), irradiance (\cite{niinemets1998}) and soil moisture (\cite{keenan2009}).  LUNA predicts the optimal partitioning of N among the maximum rate of carboxylation $V_{c,max}$, the maximum rate of electron transport $J_{max}$, and other leaf N components for a given leaf nitrogen (per unit area) and environmental boundary conditions (CO$_{2}$, temperature, humidity, soil moisture, radiation and day length), assuming co-limitation of photosynthesis by both electron capture and carboxylation processes.  $V_{c,max}$ is thus primarily controlled by the N per unit leaf area $N_{area}$ and the prevailing meteorology. The capacity of the LUNA model to represent appropriate responses to temperature and CO$_{2}$ was demonstrated by \cite{xu2012} while the global calibration and the geographical predictions of the model were described by \cite{ali2016}.

Secondly, the `FUN' (Fixation and Uptake of Nitrogen) model was incorporated into CLM5. FUN simulates the dynamics and carbon economics of N acquisition from the environment, building on (\cite{fisher2010fun} and \cite{brzostek2014} and \cite{shi2016}). A major issue confronting N cycle models is how to manage 'excess' carbon under conditions where carbon assimilation is faster than that which can be supported by nutrient uptake (\cite{dekauwe2014}). Previous versions of CLM model reduced the gross photosynthetic flux down to the level at which growth could be supported by N uptake. This process occurred after the calculation of stomatal conductance (which is linked to assimilation rates via the Ball-Berry model) and therefore created inconsistencies between C assimilation and water cycling, especially under conditions of N limitation, as discussed extensively in the literature (\cite{medlyn2011}, \cite{bonan2012}, \cite{dekauwe2014}, \cite{walker2014}). The FUN model, in contrast, employs the principle that for the alternative sources of nitrogen acquisition by plants (active uptake from soil, retranslocation from senescent tissues, and symbiotic N fixation) there is a concurrent cost in terms of carbon (\cite{bloom1985}, \cite{jiang2017}, \cite{terrer2018}) which is determined by the abundance of nitrogen in the soils and plant tissues and the temperature-dependant cost of N fixation. Within FUN, plants spend excess carbon for nitrogen uptake preferentially using pathways which are `cheaper' to them. Where N is scarce in the environment, carbon that would otherwise have been used for plant growth can be deployed to acquire (pay for) more N. Thus, N limitation leads to both reduced growth and higher expenditure of C on N uptake.   The switching of N uptake to cheaper pathways results in the preferential use of symbiotic N fixation when soil mineral N concentrations are low (\cite{vitousek2002}), whereas CLM4 and CLM4.5 predicted N fixation as a linear function of net primary productivity (\cite{cleveland1999}; \cite{wieder2015}).   The FUN model is documented within the CLM technical note at \url{https://escomp.github.io/ctsm-docs/doc/build/html/tech_note/FUN/CLM50_Tech_Note_FUN.html}. 
 
Thirdly, CLM5 contains the capacity to modify tissue C:N ratios with spatial and temporal variation in the cost of N uptake.  Plant trait databases (e.g. \cite{kattge2011}) typically indicate a wide variability in tissue N concentrations within a single plant functional type (\cite{wright2005}). \cite{ghimire2016} introduced the capacity for variable tissue C:N ratios, whereby the total nitrogen supply in each timestep is partitioned between tissues in proportion to their relative 'demand' terms (ascertained from the target C:N ratio and the carbon allocated to the pool).  The FUN model used in CLM5 was originally conceived in the context of a static tissue C:N ratio. Carbon is allocated between growth and N uptake in a manner that exactly tracks the target C:N ratio. To allow a variable C:N ratio in the context of FUN, it is necessary to modify how much C is expended on N uptake vs on growth from this default calculation. The degree to which tissue C:N content varies as N becomes limiting is unclear in the literature, and N cycle models typically include semi-heuristic and/or bounded representations of tissue N adjustments under N limited conditions (\cite{zaehle2010},\cite{ghimire2016}). Here we include a placeholder algorithm that modifies C:N ratios with 1) the environmental cost of N acquisition and 2) the degree to which the plant is already far from its target N content. When N is scarce in the environment, less C is spent on uptake, increasing C:N ratios. When plants become N limited and C:N ratios increase, however, expenditure increases. This algorithm is also documented in the CLM5 technical description ((\url{(https://escomp.github.io/ctsm-docs/doc/build/html/tech_note/index.html)}) and includes three tuning parameters ($flexcn_{a}$, $flexcn_{b}$ \& $flexcn_{c}$) the sensitivity of which is investigated here.

\subsection{Sites}
We utilized four observational flux tower sites, in order to capture a range of environmental conditions, across tropical evergreen broadleaf forest (Caxiuan\~a, -1.9$^{o}$N, -51.4$^{o}$E),  sub-alpine evergreen forest (Niwot Ridge, 40.3$^{o}$N, -105.4$^{o}$E), temperate deciduous broadleaf forest (Harvard Forest, 42.5$^{o}$N -72.2$^{o}$E) and seasonally dry evergreen needleleaf forest (Metolius Forest, 44.5$^{o}$N -121.6$^{o}$E).  Each site was driven with the available site-specific meteorological data. Soil texture, color, depth and, plant functional type composition were taken from previous model validation exercises undertaken at these sites \cite{bonan2012} for Metolius and Harvard Forest, \cite{fisher2007} for Caxiuan\~a and \cite{bonan2013} for Niwot Ridge.   We note here that our use of a small number of single sites cannot fully describe the environmental conditions under which the model is applied. Instead, our aim is to provide detailed illustrations of how model parameter choices impact a model outputs under a range of realistic forcing conditions. Conducting and analysing these simulations globally is beyond the scope of this study, but investigations of the response of global biogeochemical feedback strength to the important dimensions of parameter space identified by this investigation will provide the basis of future studies. 

\subsection{Simulation Protocol}
We spun up the default version of the model for 400 years at each site, in 'accelerated spin-up' mode (whereby the decomposition of slower cycling carbon and nitrogen pools is increased for the duration of the spin-up, and the pool sizes are modified accordingly at the end of the spin-up - see (\cite{lawrence2018}). We used pre-industrial CO$_{2}$ concentrations (274ppm) and recycled the available site-level half-hourly meteorological inputs for the duration of the simulations. 400 years was sufficient to bring all soil and vegetation carbon and nitrogen pools into equilibrium. Subsequent to this, parameters were perturbed (see below), and a second spin-up conducted for 180 years for each site, parameter and value combination, with accelerated mode turned off. Restarting from the state at the end of the perturbed parameter spin-ups, we ran 'control' transient simulations for each site and parameter combination starting in 1760 and invoking historical transient CO$_{2}$ and N deposition until 2018.

For each site/parameter/value combination, we also ran one elevated CO$_{2}$ and one elevated nitrogen deposition simulation. Following \cite{wieder2019}, who conducted global runs with the default parameterization of the model, we increased CO$_{2}$ to 550ppm, starting in 1998 and for Nitrogen, we added 5gN $m^{-2}$/year, also starting in 1998. 

\subsection{Perturbed Physics Ensemble (PPE)}
The CLM5 has a large array of parameters that might have direct and indirect impacts on biogeochemical cycling in the model.  Here we focus on an array of thirteen parameters, chosen for their direct mechanistic impacts on responses to CO$_{2}$ and N fertilization. These parameters are narrowed down from the much larger complete set of CLM5 parameters (which will not enumerate here, since calculation of this number in a model as complex as CLM5 is a scientific exercise in its own right) using a process informed primarily by the model structure, as well as the extensive iterative investigations of the CLM5 model during the development process.  We focus here on biogeochemically important inputs. Parameters relevant to the hydrology and surface energy balance processes (with the exception of the stomatal slope) since sensitivity to these will be assessed by studies currently underway, and parameters of the new plant hydraulics scheme are discussed in \cite{kennedy2019}.

To reduce the complexity involved in understanding and attributing changes in model behaviour with parameter variation, we conducted one-at-a-time (OAAT) perturbations of the parameters around the default state, which (compared to a global parameter sensitivity analysis) allows more straightforward visualization and interpretation of the results.  At each site, we conduct simulations with five values of each parameter (one of which is the default) across the ranges described below. We recognize that a global parameter sensitivity analysis would also be of interest, and indeed have conducted many such analyses. We concluded that OAAT analysis provides clearer insight into the functionality of the parameters with respect to biogeochemical forcing. 

\subsection{Default parameter determination}
During the development of CLM5, we conducted extensive testing into the possibility of inverse-calibration of the model using neural network derived emulators, in conjunction with a variety of gridded global data products. We did not use the results of this effort for a variety of reasons, including,  1) the high dimensionality of the space, 2) the degree of non-orthogonal behaviour in the parameter response surfaces, 3) the tendency of the modeled ecosystem to die under lower pre-industrial CO$_{2}$ conditions when calibrated using present-day CO$_{2}$, 4) the dominance of high productivity areas over the calibration of a particular PFT at the expense of marginal areas, and the resulting effect that those marginal areas die off in model spin-up, 5) the subjectivity in weighting of the various observational data products and 6) philosophical issues concerning whether the structural validity of the calibrated model could be independently assessed, given the sparsity of validation datasets. The parameterization used in the default version of the model consequentially utilized the mean trait values where possible from trait datasets (\cite{lawrence2018}) in combination with targeted manual tuning of parameters (including allocation parameters and Nitrogen costs) that were less well constrained by data, to avoid excessively unrealistic values for some model simulated states. We recognize that this more subjective calibration method is less than ideal, and ongoing studies will focus on more limited scope calibration attempts to overcome the difficulties described above, as illustrated by \cite{fer2018}. 

\subsubsection{Parameters selected for perturbation experiment}

The parameters included in our sensitivity analysis are listed in Table \ref{table_ranges}. Their impact on model functionality is described below.  The shorthand version of the parameter name is given in italics.  

\textbf{Specific Leaf Area}, \emph{`SLA'}, is a parameter common to almost all land surface models, which determines the area of leaves (in m$^{2}$) derived from one gram of leaf biomass. High SLA values correspond to thinner, 'cheaper' leaves which lead to thicker vegetation canopies and greater interception of light. In CLM5 high SLA leaves also have lower N per unit area (given a target mass-based C:N ratio) and thus lower photosynthestic capacity per unit area.

\textbf{Fine root mass per unit leaf mass}, \emph{`froot\_leaf'}, is the (constant) ratio of fine root biomass to leaf biomass. In CLM5, fine root biomass affects the capacity of vegetation to acquire both water and nutrients via the resistance of the rhizosphere to water flow, and the cost of N acquisition in the FUN model, respectively. Lower values should in principle incur a cost to vegetation in terms of below-ground resource acquisition.  The carbon allocation in CLM5 uses a fixed-fraction approach, with constant ratios between, leaf, fine roots, stem and coarse root allocation. As such, the carbon allocation scheme is thus capable of generating physiologically sub-optimal leaf area index values, for example, with some of the canopy in negative carbon balance. Newer codes within the CLM framework (e.g. the FATES model, https://github.com/NGEET/fates, \cite{fisher2015}m \cite{fisher2018vegetation}) account for this issue by removing leaves in negative carbon balance. 

 \textbf{Stem to leaf mass ratio}, \emph{`stem\_leaf'}, controls the fraction of biomass to stem tissues as a fraction of the leaf allocation. Increases in this parameter both decrease the leaf area index (LAI, m$^{2} m^{-2}$) achievable for a given amount of (C and N) allocation to growth, and increase the equilibrium woody biomass. Woody biomass in CLM5 has no functionality and thus there is no cost to plants associated with low values of this parameter).

\textbf{Nitrogen costs}, \emph{`n\_costs'}, are a set of parameters that determine the environmental cost of Nitrogen uptake from the soils. The FUN model introduces the concept of carbon payments for active N uptake, using a bulk approach to account for the costs to plants of low environmental N availability, including; carbon expended on root exudates, symbiotic relationships with soil fungi, expenditure on fine root growth and/or increased metabolic rates to support active uptake. Therefore, uncertainties associated with the cost functions, in particular, are large. These six parameters were defined and calibrated by \cite{brzostek2014}. In CLM5, we maintain the ratios of the parameters as defined by the previous FUN model calibration, but allow the magnitudes of the parameters to change concomitantly, since the definitions of soil N availability vary between CLM5 and \cite{brzostek2014}. The six \emph{'n\_costs'} parameters include \emph{ekc\_active} and \emph{ekn\_active}, the rate constants that relate the cost of ectomycorhhizal uptake to the fine root carbon pool (\emph{ekc\_active}) and soil N concentration (\emph{ekn\_active}), respectively and their counterparts \emph{akc\_active} and \emph{akn\_active} for arbuscular mycorrhizal uptake, and \emph{kc\_nonmyc} and \emph{kn\_nonmyc} for non-mycorrhizal active uptake. 

\textbf{The fraction of carbon that can be used for fixation}, \emph{`fracfixers'}, is the fraction of the net primary production (NPP) or gross primary productivity (GPP) minus autotrophic respiration (not including carbon spent on N uptake, $r_{auto}$) that can be used for symbiotic N fixation.  It is thus a proxy for the ecosystem-level fractional productivity of symbiotically N-fixing plants. The constant \emph{`fracfixers'} parameter here acts in lieu of, for example, a prognostic model of N fixer distributions, or an input map of their abundance, both of which could plausibly be added in future generations of CLM. 

\textbf{Leaf carbon:nitrogen target}, \emph{leafcn}. CLM5 uses the concept of a 'target' leaf C:N ratio, around which flexibility is allowed, given the environmental costs of Nitrogen. As with most land surface schemes, higher leaf N concentrations lead to both higher maintenance respiration costs and higher photosynthetic capacity (in this case via the LUNA model).  

\textbf{Slope of the stomatal conductance model}, \emph{medlyn\_slope}. CLM5 replaced the Ball-Berry stomatal conductance scheme with the Medlyn stomatal conductance scheme (\cite{medlyn2011}). The slope parameter determines the degree of stomatal opening for a given combination of assimilation capacity, CO$_{2}$ and vapour pressure deficit, in the absence of plant hydraulics limitations. Thus, is it intrinsically linked to the response of ecosystems to CO$_{2}$. 

\textbf{C:N flexibility }, \emph{cn\_flex\_a, cn\_flex\_b, \& cn\_flex\_c},  as described in the CLM5 technical note, determine the response of tissue C:N ratios to depletion of N in both the environment and the plant tissues themselves. Only the thid parameter (c) had an important impact in this sensitivity analysis, and so we focus here on the impact of that parameter. 

Other variables we tested in earlier versions of this analysis included rate constants controlling the growth and maintenance respiration rates, the fraction of ectomycorrhizal fungi, and parameters 'a' and 'b' of the flexible C:N ratio model. None of these had important impacts on the fertilization responses, and growth and respiration rates affected the model state in a linear and predictable fashion. They are omitted here for the sake of simplification, but data can be accessed via the available simulations. 

\subsubsection{Parameter Ranges}
\emph{SLA} and \emph{leafcn} are readily observable and thus among the most common traits represented in the TRY plant trait database (\url{www.try-db.org}). We derive values from the analysis by (\cite{kattge2011}) for their mean and distribution. For both parameters, values are log-normally distributed. We here take the PFT average log-normalised standard deviation range from TRY and apply it across all PFTs (thus the lower bounds are closer to the mean than the higher bounds, Table \ref{table_ranges}). The range of \emph{fracfixers} and was set to vary from 0 to 100\% of ecosystem productivity being available for fixation.  \emph{medlyn\_slope} varies across the stndard errors reported in \cite{dekauwe2015}. Reasonable ranges of the \emph{cn\_flex} parameters are unknown, since these physiological controls are poorly understood. Thus, we explore an order of magnitude in each direction from the default, to obtain an understanding of the general sensitivity to this parameterization. The same issues apply to $N_{costs}$ parameters. 

\section{Results}
Our analysis used four sites, with eight parameters perturbed to four different values, each of those with control, CO$_{2}$ and N fertilization simulations.  We examine the parametric responses of five output variables chosen for their importance in the overall biogeochemical system (GPP, NPP, LAI, leaf nitrogen  ($N_{area}$, gN m$^{-2}$), and total vegetation carbon (g m$^{-2}$). We did not look at soil output variables as our parameter analysis is focused on vegetation responses. 

In this section we assess first the impacts of parametric variation on the system state and second the responses to carbon and nitrogen fertilization. We do not focus on the goodness of fit to observations, because our aim is to discern the primary parametric controls over CO$_{2}$ and N fertilization responses, and including extensive data-fitting and calibration experiments (including on parameters mostly unrelated to C and N cycling) would detract from this goal. 

\subsection{Overall system state}
The parameter responses of the site-level water and nitrogen limitation (see below) in the control (unfertilized) parameter ensemble are shown in figure \ref{btran state}. GPP, NPP, LAI, leaf N content per unit area and total vegetation carbon are shown for the Niwot Ridge, (Figure \ref{NR1 state}), Caxiuan\~a (Figure \ref{CAX state}), Harvard Forest (Figure \ref{HVF state}) and Metolius (Figure \ref{MET state}) sites. In each of these figures, the y-axis corresponds to the output variable and the x-axis represents the range of parameter perturbation with -1 to +1 corresponding to the range of variation in each parameter, depicted in Table \ref{table_ranges}. Lines of differing colors illustrate responses to variation in the thirteen target parameters, with the central crossover point in each figure corresponding to the default case.

\subsubsection{Water and N limitation across sites}
The four sites varied in the degree to which they are limited by water and N availability in the control state, with important impacts upon the parametric response characteristics (Figure \ref{btran state}). Water limitation in CLM5 can be expressed via the $B_{tran,mn}$ parameter, which is a model property calculated as a function of the prognostic leaf water potential. $B_{tran}$  is used to reduce assimilation and thus stomatal conductance by scaling the photosynthetic capacity, $V_{c,max}$. Leaf water potential and  $B_{tran}$ vary diurnally, so we use the average daily minimum value here as a metric of drought stress (1 is no drought stress, 0 is total stomatal closure). For N limitation, the percentage of total NPP (where NPP is GPP-$r_{auto}$) that is used to pay for nitrogen uptake (PNU) is a direct indicator of how N stress affects the carbon budget. 

The Caxiuan\~a site (CAX) experienced substantial drought stress and little nutrient stress ($B_{tran,mn}$ $\sim$0.6, PNU$\sim$13\%). Niwot Ridge (NWR) experienced substantial amounts of drought and N limitation, ($B_{tran,mn}$, $\sim$0.55, PNU$\sim$21\%). Harvard Forest (HVF) site had very little drought stress, and moderate N limitation ($B_{tran,mn}$ $\sim$0.96, PNU$\sim$15\%).  Metolius (MET)  had some drought stress and the least nutrient stress ($B_{tran,mn}$ $\sim$0.86, PNU$\sim$10\%).  
Thus, the sites chosen span a range of initial limiting conditions.  

Differences in the initial conditions at the sites have implications for the parametric responses. Underlying limitations by various boundary conditions (water, nutrients, light, temperature) can affect both the ability of the system to respond to fertilization and/or parameter modifications affecting other resources, as we describe further in the following sections. 

\subsubsection{Parametric control over system state}
At all sites, specific leaf area (\emph{slatop}) has a large first-order impact on LAI, and on leaf N concentrations (Figure \ref{NR1 state}), as expected, since it is used directly in the calculation of both of these quantities. For GPP, the impact of SLA is always positive, despite the higher LAI causing lower N per unit area of leaf (which reduces per-unit leaf area photosynthetic capacity). The impact of SLA on NPP can, however, be negative (e.g., Metolius), since the fixed-fraction allometry model in CLM5 can generate shaded leaves that are in negative carbon balance at high LAI. At Niwot Ridge, the parameter range straddles an optimal SLA (or local maximum at the second-largest parameter values) for NPP, again reflecting the assimilation/respiration cost trade-off.

Root:leaf ratio (\emph{froot\_leaf}) has 1) a direct effect on LAI, via changes in allocation fractions to different tissues, 2) an indirect effect on NPP, given the modified maintenance costs of the resultant fine root biomass pools, 3) an effect on the uptake of water via root biomass and 4) an effect on nutrient limitation through root-mediated N uptake costs.  The first two impacts decrease NPP with \emph{froot\_leaf} and the latter two increase it.  Both types of effect are seen here, with the Niwot Ridge and Metolius sites (which have the lowest productivity and LAI, and thus presumably very low root biomass in the first place) suffering very large reductions in GPP, NPP, and LAI at low \emph{froot\_leaf} ratios, reflecting mechanisms 3 and 4.  Harvard forest and Caxiuan\~a both show increases in all variables with low \emph{froot\_leaf}.  In Figure \ref{btran state}, we can see that low \emph{froot\_leaf} is a primary control over drought stress at Metolius, in particular. 

Low values of \emph{stem\_leaf} ratio typically resulted in much higher LAI at all sites, as well as declines in vegetation carbon, as expected from the lower stem carbon allocation. At Niwot Ridge, there is an optimum (or local maximum) stem carbon observed, due to the impacts of a too-high leaf area index, which caused declines in NPP at low \emph{stem\_leaf}. At all other sites, the impact of increasing leaf allocation is to increase NPP (almost doubling at Metolius). Decreases in stem:leaf ratio also caused a decline in leaf N per unit area, potentially on account of the higher demand for N uptake and thus higher N limitation. This effect is llustrated clearly for Niwot Ridge and Metolius in Figure \ref{btran state}). 

The first-order impact of increasing \emph{Ncosts} was to increase the respiration losses associated with nitrogen uptake.  Across the range tested here (bearing in mind that these parameters are somewhat unconstrained) there was limited impact of \emph{Ncosts} on the system state with the exception of the Niwot Ridge site, where N limitation was highest and which showed a decline in LAI, NPP, GPP and vegetation carbon when costs were high, as well as slight increases in leaf N content where costs were low. Other sites showed more muted responses in the vegetation state. 

Increasing \emph{fracfixers} affects the maximum amount of carbon that plants can pay towards symbiotic nitrogen uptake (N fixation). For all sites and variables here, it has a relatively moderate and positive impact on the control state, in particular at the Niwot and Harvard forest sites. As is the case for \emph{N\_costs}, with high fixing capacity, N uptake expenditure is lower, N uptake high, and NPP moderately increased, resulting in increases in LAI and a slight response of GPP to that LAI increase.

Changing \emph{leafcn} -the target leaf C:N ratio- over one standard deviation from the mean TRY values, had substantial impacts on all the output variables at all sites. Higher \emph{leafcn} values decrease $N_{area}$, decreasing photosynthetic capacity and hence GPP.  In particular, the higher \emph{leafcn} values had $N_{leaf}$ and photosynthetic capacities so low that GPP was almost zero at Metolius and Harvard Forest and more than halved at Niwot Ridge (\ref{NR1 state}).  We do not find much evidence of saturation in the GPP vs. \emph{leafcn} relationships, in that GPP response does not level off at high leaf N content. This is consistent with the analysis by \cite{lawrence2018} who find that V${_c,max}$ numbers in CLM5 might be low compared to observations.  For NPP,  decreases in respiration rates caused by high \emph{leafcn}, and reduced carbon expenditure on N uptake (Figure \ref{btran state}) somewhat ameliorated the GPP increases, particularly at Caxiuan\~a, which has high ambient temperatures and respiration rates as a result. At Niwot, an optimum \emph{leafcn} ratio (or a local maximum in the parameteric response) was apparent. For NPP this was around the default value, but for LAI it was 0.5 standard deviations lower, reflecting the complex interactions between the expenditure on N uptake and acquisition of N for tissue growth. 

Changes in \emph{medlyn\_slope}, the stomatal slope parameter, most directly impact GPP, but can also affect leaf temperature via changes in canopy resistance and thus sensible heat flux. Water limitation is also modified as a result of altered transpiration. At Caxiuan\~a (Figure \ref{CAX state}), the very lowest values of  \emph{medlyn\_slope} caused a complete death of the ecosystem. In Figure \ref{btran state} we can see that low values of the Medlyn stomatal slope ameliorated plant drought stress (by reducing stomatally-mediated water losses), presumably at the cost of limiting photosynthesis past the point at which positive plant growth could be maintained. At Niwot, a clear local maximum was indicated in the response to the Medlyn parameter, indicating that above this level, stomata lose too much water and below it, they do not photosynthesize enough. At Metolius and Harvard forest, only the lower \emph{medlyn\_slope} values affected the photosynthetic rates, indicating co-limitation of GPP by processes other than stomatal conductance.  

 \emph{cn\_flex\_c} affects the degree to which plants attempt to maintain tissue C:N values near to their target when environmental costs are varying. For the state impacts here, the role of this parameter was limited, but for the step-change fertilization it was much more important, so we will discuss it's role further in the section below. 

\subsection{CO$_{2}$ and Nitrogen deposition responses}


\subsubsection{Timescale dependance of responses to fertilization}

The initial changes in carbon economy induced by the fertilization experiments are, where conditions permit, amplified by increases in the leaf area index.  Figure \ref{timescales} illustrates, for the default parameter set, the evolution through time of output variables (GPP, NPP, LAI, $B_{tran,mn}$, NPP spent on N uptake). These results illustrate the complexities of the response. 

For example, the initial stimulation of GPP by high CO$_{2}$ ($\sim$1.37) lead to a $\sim$50\% increase in LAI. This, however, lead to increase drought stress (lower $B_{tran,mn}$), and hence no large increase in GPP. At Metolious, in contrast, initial GPP stimulation was similar, the resulting growth in LAI was much slower, given the lower overall carbon turnover at this colder site. The 50\% increase in  carbon at 80 years had, however, a much larger impact on GPP, giving a stimulation of ** at 80 years. 
 At Harvard forest, the initial stimulation in the first year was lower (25\%) potentially on account of the higher carboxylation capacity at this site (as illustrated by the lower response to decreasing leaf C:N ratios in figure ** than other figures).  At all sites, the fraction of NPP spent on N uptake increased, but most dramatically at Metolius, where initial N expenditure was lowest initially on account of low growth rates. 
 
As expected from the initial conditions, Niwot Ridge responded most strongly to nutrient additions. An initial NPP stimulation of 1.3 for NPP was magnified by subsequent increases in LAI to nearly 1.8 times the control value, causing a stimulation of 1.7 in NPP. None of the other sites exhibited such a dramatic change. In particular, Caxiuana, which was heavily water limited in the first instance, adding nitrogen only resulting in stimulation in LAI, GPP and NPP of around 10\%. Stimulation at Metolious was higher than Caxiuana, despite it's lower initial N limitation, given it's lack of limitation by water.

The time-dependence of the responses generates an interesting issue for interpretation, since where LAI increases a lot, the responses tend to be dominated by this feature of the response. The first year responses, by way of contrast, tend to be much more similar across sites. 




Figures \ref{GPP CO2 and N respones 2001}, \ref{NPP CO2 and N respones 2001} and \ref{LAI CO2 and N respones 2001} illustrate the responses to both N and to CO$_{2}$ fertilization simultaneously, for GPP, NPP and, LAI. Corresponding figures for vegetation carbon and $N_{area}$ are in the supplemental information (SI Figures 1 and 2). Figures for each site showing the individual N and CO$_{2}$ responses are also included in the SI (Figures 3 to 10).

We calculated the impacts of the CO$_{2}$ and N deposition by assessing the ratio of the fertilized to control simulations, averaging years 15-19 after fertilization began. For some parameter values, the constraint on growth placed the control system in a state close to or at death (zero live carbon pools). Addition of N or CO$_{2}$ therefore caused either a very strong recovery of the plants, or no recovery at all where biomass was already zero, leading to some large non-linearities in the relationship between parameters and environmental forcing. To focus on the responses of parameter combinations that give more reasonable initial vegetation conditions, parameter combinations that caused LAI to be less than 60\% of the control simulation value were excluded from the fertilization plots.

\subsubsection{Differences across sites}
The responses to CO$_{2}$ and Nitrogen fertilization were substantially different between sites and across parameter ranges. 

For the default parameterization (the intersection of all lines in figures \ref{GPP CO2 and N respones 2001} to \ref{LAI CO2 and N respones 2001}), the impacts of CO$_{2}$ fertilization on  GPP (1.44; CAX, 1.44 NWR, 1.24 HVF, 1.45; MET) were fairly constant across sites, but for NPP (1.66; CAX, 1.28; NWR, 1.22; HVF, 1.62; MET) the sites with higher responses had higher N limitation in the control cases.  LAI responses (1.47; CAX, 1.24 NWR, 1.25; NVF, 1.27 MET) reflected the NPP change, except for at Metolius where the canopy did not respond to the higher NPP. 

For Nitrogen responses, while impacts on  GPP were all slightly positive, (1.02; CAX, 1.04; NWR, 1.12; HVF, 1.13 MET), they were lower for the drought-stressed Caxiuan\~a and Niwot sites.  For NPP (0.96; CAX, 0.98; NWR, 1.25; HVF, 1.06 MET), increasing $N_{leaf}$ (1.04; CAX, 1.01; NWR 1.02 NVF; 1.01 MET) caused increasing respiration and negative impacts on NPP at some sites, even though impacts on LAI (1.06; CAX, 2.09 NWR, 1.21; HVF, 1.12 MET) were all positive, reflecting the lower costs of N uptake and greater capacity to acquire N to build tissues and retain carbon for growth.  

\subsubsection{Impact of parameters}
Across the four sites, parametric impacts on fertilization responses were very varied, when compared to the relatively consistent impact of parameters on the initial system state. A majority of parameters show no consistent relationship with CO$_{2}$ or N deposition responses across sites, and the range of their impacts varied over an order of magnitude. Thus, the parameter response space of the model is complex with regard to these changing initial conditions, environments, and drivers on global change. Again, we go through the parameters one at a time to highlight the primary features of the model response and how they relate to expectations based on model structure. 

The strong impact of \emph{slatop} had important implications for the initial state in many cases, and for several of the lower values, we screened out the results of the fertilization on account of very low LAIs (which resulted in erratic fertilization responses). The impact of \emph{slatop} on fertilization was different at every site. For example, at Metolius, high  \emph{slatop} was related to high N responses, plausibly since these cheaper leaves could produce a greater LAI/GPP feedback in response to the N fertilization. At Niwot Ridge, high SLA was related to strong GPP responses to CO$_{2}$. These simulations have higher water limitation (on account of their larger canopies, figure \ref{btran state}) and so respond more vigarously to the increased water use efficiency of high CO$_{2}$ growth conditions (while reducing their NPP and LAI with N fertilization, on account of higher respiration costs). These contrasting results are illustrative of the complexity of the interactions between model initial state, environmental forcing and parameterization, and of the need to understand how multiple simultaneous growth limitations can influence responses to climate forcing.   

Despite its strong impact on the system state,\emph{froot\_leaf} had very little impact on fertilization responses, except in the case of very low values at Metolius, where it increased the CO$_{2}$ responsiveness, and at Harvard, where low values decrease both responses. At Harvard the low root allocation runs had higher initial LAI, and thus the responsiveness of the system to fertilization is lower since the feedback from LAI to GPP (which is a non-linear response related to the saturation of light interception). 

Low values of \emph{stem\_leaf} ratio caused lower responses to fertilization for GPP and LAI at Caxiuan\~a and Harvard Forest (Figures \ref{GPP CO2 and N respones 2001} and \ref{LAI CO2 and N respones 2001}). Initial LAI was larger in these cases, explaining the decrease in fertilization response as above (lower increases in GPP for a unit change in LAI).  At Niwot Ridge, low values of \emph{stem\_leaf} were associated with very high LAI and high water limitation (Figure \ref{btran state}), and CO$_{2}$ fertilization thus had a large impact on GPP and NPP (as for \emph{slatop}), but the high NPP has no impact on LAI (SI Figure 3), reflecting the still-high N limitation.  Correspondingly, adding Nitrogen in the low \emph{stem\_leaf} runs increases LAI substantially, but only has a very limited impact on GPP and NPP, given water limitations.

The \emph{N\_costs} parameterization primarily impacts the respiration expended on N uptake and the availability of carbon for tissue growth thus, its impact is greatest for LAI (Figure \ref{LAI CO2 and N respones 2001}). At Caxiuan\~a, Niwot and Harvard forest, high  \emph{N\_costs} resulted in stronger LAI responses to N fertilization, since these cases had higher N limitation in the first instance (Figure \ref{btran state}).  These high \emph{N\_costs} scenarios also showed a slight increase in the CO$_{2}$ fertilization response, perhaps since their high N limitation was alleviated by extra assimilated carbon.   Low values of \emph{N\_costs} were associated with low CO$_{2}$ responses (and vice versa) in particular at Caxiuana and Harvard Forest, which can be interpreted as a reflection of the lower C expenditure on N uptake in the control simulations.  While nitrogen uptake costs are important to the fertilization responses, they do not affect the total amount of N in the system, and therefore appeared to play a secondary role to the amount of N fixation available.

The fraction of N fixers (\emph{frac\_fixers}) has a particularly dominant role over the model response to step changes in N and to CO$_{2}$. For GPP and LAI, all sites displayed a 'trade-off', whereby high N fixer systems responded strongly to CO$_{2}$ but not N, and low fixer systems responded strongly to N but not CO$_{2}$ (Figure \ref{GPP CO2 and N respones 2001}, \ref{LAI CO2 and N respones 2001}).  For NPP, N deposition responses were muted at Metolius, and reversed in sign at Caxiuan\~a, where the low N fixer cases had a very negative (-8\%) NPP response (despite a +7.6\% GPP response), on account of the model increasing the N content of the leaves (and thus respiration rates) by up to 10\% in the lowest N cost case (SI Figure 6). Nonetheless, given the lower N uptake costs in the fertilized N experiment, leaf area was substantially increased (up to 14\%) in these cases too.  At Niwot Ridge the changing fixation capacity had no impact on the response of GPP to N fixation, despite its influence on LAI, suggesting that water limitation was a greater constraint on total GPP than leaf area.  At Metolius, again the changing fixation capacity had no impact on the response of GPP to N deposition (Figure \ref{GPP CO2 and N respones 2001}) but in this case, it had no impact on LAI either. Only \emph{slatop} and \emph{stem\_leaf} had large impacts on N fertilization at this site, suggesting fundamental metabolic limitations to canopy carbon economics (e.g. a short growing season) that render only higher LAI simulations sensitive to fertilization. 

The impact of target \emph{leafcn} on fertilization responses was not particularly dominant (recalling that this was a significant control over the system state at all sites), an exception being Niwot Ridge, where low values of \emph{leafcn} had very strong to CO$_{2}$ fertilization (the strongest response at this site for GPP). These cases suffered from high water limitation, given their high $N\_{leaf}$ values and high gas exchange fluxes (Figure \ref{btran state}) and thus CO$_{2}$ fertilization alleviates this problem. At Caxiuan\~a, high \emph{leafcn} cases had stronger responses to N fertilization, but responses to N fertilization were low at this site in general.  

\emph{grperc}, or the degree to which growth is 'taxed', is constant and not responsive to any of the physiological processes upstream that depend on fertilization, thus, at most sites, its impacts are minimal, or indirect. 

At Niwot Ridge, runs with low values of the \emph{medlyn\_slope} parameter (and thus high water use efficiency) were able to respond very strongly to increasing Nitrogen (SI Figure 4), given their lower initial water stress (Figure \ref{btran state}). Surprisingly, however, given its critical role in the responsiveness of stomata to CO$_{2}$, we did not find a widespread impact of the Medlyn parameter on the fertilization response, potentially suggesting that the impact on the state condition had already accounted for the impact of this parameter. 

In general the impacts of the maintenance respiration intercept \emph{lmr\_intercept} were small, and limited to the impact of large values increasing fertilization responses to CO$_{2}$ and N, potentially on account of the low LAI and thus greater feedback between LAI and GPP, and in line with its limited impact on the system state, \emph{frac\_ectomy\_fungi} had no discernible impact on fertilization responses. 

There was limited impact of \emph{cn\_flex\_a} on the fertilization responses, as per it's impact on the state.  At Harvard Forest, low values of  \emph{cn\_flex\_b} increased responses to N deposition and decreased response to CO$_{2}$. Leaf N concentration followed suit, increasing substantially in the N deposition experiment. Low values of \emph{cn\_flex\_b} decrease the amount of carbon spent on N uptake when \emph{N\_costs} are high, therefore potentially suppressing leaf N content. Adding N overcomes this suppression, leading to greater fertilization responses.

The impact of \emph{cn\_flex\_c} (the parameter controlling the degree to which plant C:N ratios are constrained to their target) appeared to have its strongest impact on the fertilization responses of vegetation carbon (SI Figure 2)  at Harvard Forest and Niwot Ridge (and Caxiuan\~a, to a smaller extent). These impacts do not mirror those on leaf area index or NPP, and overall the response to this parameter appears complex.  At Metolius and Niwot, \emph{cn\_flex\_c} has a dominant control over the impacts of CO$_{2}$ fertilization on $N_{leaf}$ (SI Figure 1).  The impacts on vegetation carbon are likely due to the larger decreases in tissue N content that are permissable when values of this parameter are high (e.g. SI Figure 3).  Low  $N_{leaf}$ values permit higher leaf area and vegetation carbon, but reduce assimilation rates, as illustrated at Harvard Forest and Niwot Ridge (SI Figure 5 \& 7).  

\subsection{Low leaf area investigations}
In general, we found that leaf area indices predicted by the CLM5 default parameter set were often low  compared to published estimates for these sites. (for Caxiuan\~a, 3.0 (modeled) vs. 5.3 (observed) \cite{fisher2007}, for Niwot Ridge, 1.7 vs. 4.2 \cite{bowling2009}, for Harvard Forest 2.6 vs. 3.5 \cite{williams1996} and for Metolius, 0.5 vs. 1.5 (\cite{spadavecchia2011}). This is a somewhat surprising result, given the generally good agreement of globally-driven model output with LAI reported by \cite{lawrence2018}, and may plausibly reflect differences between the use of reanalysis products used to drive the model and site-specific meteorological data. 

To interrogate this further, we ran an entire separate set of simulations with higher allocation to leaves (generated from lower stem allocation parameters). These simulations indicated that, with the resulting higher, and more consistent with observed, LAI values, the model was less sensitive to both CO$_{2}$ and N deposition. In the default parameterization CO$_{2}$ and N fertilization increased LAI, amplifying these magnitude of ecosystem responses to these perturbations. For example, at Harvard Forest, the high leaf allocation simulation (LAI of 4.6, vs 2.6 for the default parameters) had a lower CO$_{2}$ response (1.18, 1.17 and 1.22 for GPP, NPP and LAI, respectively, in the high LAI case vs. 1.43, 1.66, 1.47 in the control) and a lower N response as well (1.05, 1.20 and 1.15 for GPP, NPP, and LAI, respectively, in the high LAI case vs. 1.1, 1.26, 1.21 for the control). Similar results were found at all other sites.  Thus, by virtue of their low LAI, the overall fertilization rates at these sites may be substantially larger than those in the model driven with reanalysis meteorology, and this may explain the large fertilization responses found here compared to those of \cite{wieder2019}. 

\section{Discussion}

The overall responses of the land surface components of Earth system models to CO$_{2}$ fertilization are the subject of intense investigations, and vast quantities of computing time will be expended on the single release version of each model, under the auspices of the various CMIP6 model inter-comparison projects and associated activities (\cite{meehl2014}, \cite{eyring2016}). Throughout the CMIP activities, the carbon-cycle feedbacks reported by each modeling group will be from a single default instance of the parameter space.

At the time of writing, we do not yet know the climate feedback strength of the default version of CLM5/CESM2 compared to its predecessors, but in offline simulations forced with a data atmosphere, CLM5 shows a stronger response to elevated CO$_{2}$ than previous versions of the model (Wieder et al. 2018).  The trade-off between responses to N fertilization and CO$_{2}$ response when the fraction of N fixing plants varies mirrors the change in fertilization response between CLM4, CLM4.5 and CLM5, and the CLM5 response is consistant with the capacity of the model to increase fixation rates in response to higher carbon dioxide levels. The number of degrees of freedom illustrated here precludes, however, proof that this is the only contributing factor. This analysis indicates that global experiments with lower or zero \emph{frac\_fixers} would help to ascertain whether this is indeed the case.

The CLM5 model has a number of new prognostic elements, which allow predictions in elements of the C and N cycles which were assumed to be constant in previous versions of the model. Exploration of the parameter space that controls these new elements allows a clear exploration of the sensitivities of the output space of the model to critical assumptions. 

The model output space is notable complex, and in many cases the sign of the impact of the different parameters changes between sites, and trade-offs in the influence of carbon and nitrogen economic parameters interact with both the initial state and environmental forcing. That the parameter space of CLM5 is complex is not surprising, given its relatively high process fidelity, but a) this type of sensitivity analysis,  illustrating how different parameterizations respond to different growth conditions and environmental forcing, is relatively rare for models of this kind,  b) the large degree of variability in the responses between sites and output variables will likely hinder attempts to generate land surface models with globally optimized parameters, and this analysis should help inform and guide the design of such efforts and c) the results can in general be interpreted as being in line with the expected dynamics of the model, given sufficient output data and consideration of the dynamics of the individual sites. Sensitivity tests across broader geographical ranges would help to eludicate more clearly how changes in baseline water and N limitation impact parameter responses and patterns of responses to fertilization.

Further, our results indicate that, in line with our initial hypothesis 2, model calibration efforts that only take into account parameters that affect the control model state, are unlikely to constrain the major axes of responses to CO$_{2}$ and N fertilization, since for each site there was substantial lack of overlap between the parameters controlling the default state and the fertilization responses.  We did observe, however, many instances where simulations with high leaf area index responded less strongly to fertilization, and thus the numerous parameters impacting this feature of the initial model state (dominated by tissue allocation ratios and \emph{slatop}) would also have an indirect impact on fertilization.  Our results suggest a more general point, that high LAI systems should respond less to fertilization. We did not find a clear statistical link between initial LAI and model output across the whole parameter space, but interpretation of global patterns in terrestrial cardon storage should likely take this mechanism into account. 

In the CMIP6 generation of land surface models, it is expected that a much greater fraction of model submissions will contain active representations of the terrestrial nitrogen cycle. We propose here that understanding the model parameterizations and structural decisions that contribute to the relative fertilization responses is critical to understanding the implications of the range of results obtained for all models, and that these types of analysis would allow a deeper understanding of land surface model outputs and behaviour. Many further analyses are possible, including global feedback analyses with alternative parameterizations, calibration of critical parameters against manipulation experiments, wider assessment against greater ranges of flux tower data, and tighter constrains of the parameter ranges used. We hope that this initial assessment of the CLM5 parameter space will provide a starting point for such activities. 

\subsection{The status of the CLM5 representation of plant carbon and nitrogen representation}
Our efforts to improve the fidelity of the nitrogen cycle in the CLM5 were primarily motivated by the need to move the model behaviour closer to several key dynamic features of the system. Notably, the dynamics of leaf nitrogen content with respect to environmental scarcity (\cite{lovelock1998}, \cite{brzostek2014}), the dynamics of photosynthetic capacity with respect to environmental conditions (\cite{xu2012}, \cite{ali2016},\cite{rogers2017}, \cite{bloomfield2018}), the changes in N fixation rate with vegetation state and the scarcity of N (\cite{vitousek2002}), and to include a means of deploying carbon as a means to acquire additional nitrogen (\cite{terrer2018}). The latter acts as a proxy for the numerous ways in which plant symbiont systems might adapt to low nutrient conditions, including increases in fine root exudation and allocation, fungal biomass, tissue metabolic rates and turnover/nutrient foraging.  In doing so, we integrate a suite of physiological processes that lack complete mechanistic and quantitative understanding.  By investigating the critical roles of these processes in the model dynamics, we hope to highlight a research need, and to stimulate research on and improvements in the representation of these processes.

In analyzing the results of this study, for example, we found numerous instances where the model reduced carbon use efficiency (CUE, defined here as net primary productivity, NPP, divided by gross primary productivity, GPP, where the respiratory carbon used for N uptake is not counted within NPP and thus lowers CUE) by changing tissue C:N ratios in response to forcing. In principle, there exists an optimal leaf C:N ratio that maximizes leaf C export under a given set of conditions. Despite its optimal allocation of leaf N \emph{within} the leaf (to photosynthetic enzymes) via the LUNA model, CLM5 does not include any scheme that ensures that overall tissue CN ratios conform to any physiological optima. Instead, a heuristic model adjusts tissue CN ratios round the target ratio (an input parameter).  Numerous models of this type have been proposed in the plant optimal functioning literature (\cite{vanwijk2003}, \cite{mcmurtrie2011}, \cite{anten2011} \cite{franklin2012}, \cite{mcmurtrie2013}, \cite{thomas2014}). We chose not to include this feature in CLM5 on account of limited evidence for its statistical strength in analyses of the global leaf trait datasets used in  LUNA, but future nitrogen cycling model development efforts could re-visit this topic. 

Another area requiring focus is the dominance of the 'fraction of fixers' over fertilization responses. \emph{frac\_fixers} is a proxy for the community level composition of nitrogen fixing plants. Technically, because plant density is not represented in the CLM5, this parameter represents the fraction of net assimilated carbon ($gpp$ - $r_{auto}$) that can be used for symbiotic N fixation. Ideally, this feature of the model would be prognostic, and the fraction of N fixing plants would respond to changes in environmental conditions and the relative competitive abilities of N fixing plants. In general, Nitrogen fixers are known to primarily exist in early successional parts of ecosystems (\cite{vitousek1989}), and therefore representation of their ecological dynamics would likely include a representation of vegetation demographics and succession (\cite{fisher2018vegetation}, \cite{trugman2016climate}). The 0-1 range of \emph{frac\_fixers} used here is ecologically unlikely, given the costs borne by plants that can fix N give them a competitive disadvantage in systems no longer limited by N. However, it suggest that a high CO$_{2}$ world would likely provide much greater benefits to N fixation, and that those benefits would be ameliorated by ongoing increases in N deposition.

\subsection{ Methodological Limitations}
\subsubsection{Limitations of step-change experiments}
Here we investigate the parameter space of the model using step-change experiments. These numerical experiments have the advantage that they can be informed by field experiments (\cite{wieder2019}, however, such experiments necessarily involve unrealistically rapid changes in environmental boundary conditions, and cannot necessarily be considered as analogous to either real-world or transient future scenario conditions. Further, we expect that the carbon and nitrogen cycles may take some time to adjust to the new conditions, given slow time-scale processes associated with decomposition of litter with change C:N ratios, and feedbacks between increasing vegetation biomass and soil biogeochemistry.

To test the impact of time on the fertilization responses, we also analyzed the impact of fertilization at 5 and 50 years into the elevated CO$_{2}$ and N experiments. At 50 years, the impacts of fertilization on GPP, NPP and LAI were similar, (result not shown) to those at 15 years.  However, at 5 years into the fertilization in the N limited sites (NWR and HVF), much more significant impacts of N fertilization were apparent (a x~2.06 impact on NPP, compared to ~0.95 in the year 15 analysis at Niwot Ridge, indicating that the results described above had already been subject to progressive N limitation, and were closer to an equilibrated state.

The development of the CLM5 slightly pre-dated the operational usage of the International Land Model Benchmarking Project (ILAMB) package \cite{collier2018}, \cite{lawrence2018}) for assessing model skill across a very broad range of model components.  In principle, integrating more targeted datasets and data products should narrow the range of model parameter sets that are acceptable in comparison with the suite of available and relevant datasets, as well as accelerate the rejection of model structures that are unrealistic. Ideally, this effort should include synthesis of ecosystem manipulation experiments that provide altered boundary conditions in real life, given the large differences observed here between the sets of parameters affecting present day conditions and those that impact transient responses.

Lastly, we recognize the importance of other limiting nutrients, particularly phosphorus, in controlling the influence of CO$_{2}$ fertilization (\cite{reed2015}). Phosphorus dynamics were not integrated into this version of the CLM, but alternative configurations of the CLM exist (e.g. \cite{yang2014}, \cite{zhu2016}) that will provide contrasting biogeochemical predictions in future studies.  

\subsubsection{Temperature and water responses}
In this study we investigate the impacts of parametric uncertainty on the carbon and nitrogen responses of the CLM5. In particular, we probe a targeted parameter set comprising those parameters that determine the impacts of CO$_{2}$ and N fertilization. The full response of the model to transient climate forcing has considerably more dimensions than these two transient properties. In particular, here we have not considered responses to temperature nor to changes in the supply or demand of water. The suite of parameters that in principle control temperature responses of the model are somewhat distinct from the parameters considered here, given the large number of processes that are directly or implicitly affected by temperature (photosynthesis, respiration, soil decomposition, N mineralization, cryosphere interaction) and should be the subject of further investigations. 

\section{Conclusions}
In this paper we present an analysis of a suite of parameters relevant to vegetation-mediate biogeochemical feedbacks in the extensively modified nitrogen cycle of the Community Land Model, version 5. We illustrate how responses of the model to elevated carbon dioxide and nitrogen deposition might be impacted by parameter choice. We find, in particular, that parameters controlling the degree to which ecosystems can fix additional nitrogen have a dominant effect over responses to fertilization, but limited impacts on the control state. Further, we find that the impact of most parameters is complex and affected by the balance of water, nutrient and other limitations to growth specific to individual sites

It is common practice in land surface modeling literature to present studies that focus on the modification of single aspects of model structure, process representation or calibration, and to highlight the significance of individual modifications when scaled up to global land-mediated climate feedback processes. In this analysis, we illustrate the degree to which a multitude of individual parameters can potentially have important and interacting impacts on model results, how these results differ across sites and output variables and with initial vegetation conditions.  Future studies of process and parameter modification in CLM5 might use these results as a guide to the relative (rather than absolute) importance of individual focus parameters and processes. We thus hope to move forwards from studies that modify individual processes in isolation to those that take a more holistic view of the complexity inherent in modeling coupled land surface systems.

We have attempted here to illustrate and understand the connections between the structure and parameterization of the CLM5 and its biogeochemical functionality. This necessarily complex exercise represents one small part of the assessment of CMIP6-era representations of the land surface needed to understand how advances in model structures alter our predictions of land surface functionality.  The challenges of comprehending, measuring and constraining the parametric and structural uncertainty of increasingly complex process-based and land surface models are daunting. We argue that in order to make progress on this task, the scientific community should move beyond the paradigm of assessing single points in model parameter space in great depth, ignoring all other plausible instantiations. In atmospheric sciences, for example, 'internal variability' caused by initial conditions is now routinely sampled, given its major impact on even long timescale forecasts (\cite{kay2015}). \cite{bonan2018} illustrate how initial conditions are unimportant for carbon cycle prediction on land, relative to structural and parametric uncertainty. Given this, sampling the important dimensions of parametric uncertainty in global Earth system model predictions should be taken seriously during future assessments of biogeochemical feedback strength.

\section{Acknowledgements}
RAF, WRW, BMS, KWO, and DML were supported by the National Center for Atmospheric Research, which is a major facility sponsored by the National Science Foundation under Cooperative Agreement No. 1852977. The CESM project is supported primarily by the National Science Foundation.  Computing and data storage resources, including the Cheyenne supercomputer (doi:10.5065/D6RX99HX), were provided by the Computational and Information Systems Laboratory (CISL) at NCAR. WRW was supported by the US Department of Agriculture NIFA Award number 2015-67003-23485 and NASA Interdisciplinary Science Program award number NNX17AK19G. DML was supported in part by the RUBISCO Scientific Focus Area (SFA), which is sponsored by the Regional and Global Climate Modeling (RGCM) Program in the Climate and Environmental Sciences Division (CESD) of the Office of Biological and Environmental Research in the U.S. Department of Energy Office of Science. The material from JBF is based upon work supported by the U.S. Department of Energy, Office of Science, Office of Biological and Environmental Research, terrestrial Ecosystem Science under Award Numbers DE-SC0008317 and DE-SC0016188. JBF carried out part of the work at the Jet Propulsion Laboratory, California Institute of Technology, under a contract with the National Aeronautics and Space Administration. California Institute of Technology. Government sponsorship acknowledged. JBF was supported in part by the NASA CARBON program. AW is supported by Oak Ridge National Laboratory, which is operated by UT-Battelle, LLC, under contract DE-AC05-00OR22725 to the United States Department of Energy. CX is supported by the United States Department of Energy (US DOE) Office of Science Next Generation Ecosystem Experiment: Arctic project and the UC-Lab Fees Research Program Award 237285.

\section{Data Availability}
Data from these simulations will be made available, prior to publication, on the NCAR-DASH data repository, \url{https://www2.cisl.ucar.edu/dash}

\nocite{*}
\bibliography{aguCLM5_bibtex}
\clearpage


\begin{table}
\begin{center}
\begin{tabular}{ |c|c|c|c|c|c| } 
 \hline
 Parameter Name & units & s1 &s2 & s3 & s4\\
  \hline
 slatop & m2/g & 0.6761 & 0.8769 &1.2785 &1.4791\\ 
 froot\_leaf & gC/gC &  0.342 &0.671 &1.05 & 1.1\\
 stem\_leaf  & gC/gC & 0.3057 &0.61 &1 &1\\ 
 n\_costs    & gC/gN &10$^{-2}$ &10$^{-2}$&  10$^{1}$& 10$^{2}$\\
 fracfixers  & - & 0 &0.5 & 2 & 4 \\
  grperc  &  - &0 & 0.5& 2 & 3\\
  leafcn  & gC/gN &0.7413 & 0.8932 & 1.1970  & 1.349\\
  
     medlyn\_slope  &- &0.5258 & 0.7191 & 1.1057  &1.29\\
      lmr\_intercept &log(nmol CO$_{2}$ m$^{2}$ s$^{-1}$) &0.8539 & 0.9531 & 1.1512& 1.25\\
      frac\_ecto &- & 0 &0.5& 1 & 1 \\
      cn\_flex\_a & &0 &0.5 & 2  &4\\
      cn\_flex\_b & &0.1 & 0.5 & 2  & 4\\
      cn\_flex\_c & &0.001 & 0.1 & 10 & 100\\
\hline
\end{tabular}
\end{center}
\caption{Parameters, units and ranges (as multipliers of the original values) in this sensitivity analysis .}
\label{table_ranges}
\end{table}

\section{Figures}
\begin{figure}[h]
     \includegraphics[width=1.2\textwidth]{matlab/figures/btran_npp.png}
     \caption{Response of drought index ($B_{tran,mn}$) and the \% of NPP spent on Nitrogen uptake to parameter perturbation across -1 to +1 range of parameter variation at the Caxiuan\~a (CAX), Niwot Ridge (NWR), Harvard Forest (HVF) and Metolious MET) sites. Legend colours as for Figure \ref{NR1 state}}
     \label{btran state}
 \end{figure}
 
 \begin{figure}[h]
     \includegraphics[width=1.2\textwidth]{matlab/figures/NOVc_STATE_1CLM50defpft_trans_1x1pt_US-NR1_ens_MIC_y1_2012.png}
     \caption{Response of control model states and fluxes to parameter perturbation across -1 to +1 range of parameter variation (Table \ref{table_ranges}) at Niwot Ridge.}
     \label{NR1 state}
 \end{figure}

 \begin{figure}[h]
     \includegraphics[width=1.2\textwidth]{matlab/figures/NOVc_STATE_1CLM50defpft_trans_1x1pt_Br-cax_ens_MIC_y1_2012.png}
     \caption{Response of control model states and fluxes to parameter perturbation across -1 to +1 range of parameter variation (Table \ref{table_ranges}) at Caxiuan\~a National Forest}
     \label{CAX state}
  \end{figure}
 
 \begin{figure}[h]
     \includegraphics[width=1.2\textwidth]{matlab/figures/NOVc_STATE_1CLM50defpft_trans_1x1pt_US_Ha1_ens_MIC_y1_2012.png}
     \caption{Response of control model states and fluxes to parameter perturbation across -1 to +1 range of parameter variation (Table \ref{table_ranges}) at Harvard Forest}
     \label{HVF state}
 \end{figure}
 
 \begin{figure}[h]
     \includegraphics[width=1.2\textwidth]{matlab/figures/NOVc_STATE_1CLM50defpft_trans_1x1pt_US-Me2_ens_MIC_y1_2012.png}
     \caption{Response of control model states and fluxes to parameter perturbation across -1 to +1 range of parameter variation (Table \ref{table_ranges}) at Metolius Forest}
     \label{MET state}
 \end{figure}

 
     
 \begin{figure}[h]
     \includegraphics[width=1.35\textwidth]{matlab/figures/NOVc_CNdep_GPP1__p2012.png}
     \caption{Influence of parameteric variation on the model response (fertilized/control) to 15 years of CO$_{2}$ and N fertilization on gross primary productivity (GPP), across the Caxiuan\~a (CAX), Niwot Ridge (NWR), Harvard Forest (HVF) and Metolious MET) sites. }
     \label{GPP CO2 and N respones 2001}
  \end{figure}
 
 \begin{figure}[h]
     \includegraphics[width=1.35\textwidth]{matlab/figures/NOVc_CNdep_NPP1__p2012.png}
     \caption{Influence of parametric variation on the model response (fertilized/control) to 15 years of CO$_{2}$ and N fertilization on net primary productivity (NPP), across the Caxiuan\~a (CAX), Niwot Ridge (NWR), Harvard Forest (HVF) and Metolious MET) sites.}
     \label{NPP CO2 and N respones 2001}
  \end{figure}
 
  \begin{figure}[h]
     \includegraphics[width=1.35\textwidth]{matlab/figures/NOVc_CNdep_TLAI1__p2012.png}
     \caption{Influence of parametric variation on the model response (fertilized/control) to 15 years of CO$_{2}$ and N fertilization on leaf area index (LAI), across the Caxiuan\~a (CAX), Niwot Ridge (NWR), Harvard Forest (HVF) and Metolious MET) sites.}
     \label{LAI CO2 and N respones 2001}
  \end{figure}

\end{document}

